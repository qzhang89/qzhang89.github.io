\chapter{Summary}

If your friend asks you, ``What is multiplet?'' A short answer is,
``Multiplets are the many-electron eigen-states in atoms.'' But probably
he won't be satisfied since he knows the name but doesn't really understand
the problem. Then you give him the following box and two electrons,
\begin{equation*}
\begin{array}{c|c|c|c|}
\multicolumn{1}{c}{} & \multicolumn{1}{c}{\phantom{,}1\phantom{,}} & \multicolumn{1}{c}{\phantom{,}0\phantom{,}} & \multicolumn{1}{c}{-1} \\ \cline{2-4}
\uparrow &  &  &  \\ \cline{2-4}
\downarrow &  &  &  \\
\cline{2-4}
\multicolumn{1}{c}{}
\end{array}
\quad \text{and} \quad
\bullet \ \bullet
\end{equation*}

\vspace{-1.5em}
and ask, ``Let's put these two electrons into this $p$ shell.
In which configuration do you think this system has the highest energy?''
(don't ask for the lowest one, since it can be known easily from
Hund's rule) If he complains there is no difference in the way of putting
electrons, then he is speaking mean-field language, which is exactly what
we assumed in our Chapter~\ref{ch:3}. Fortunately, your friend is convinced
that electrons with different orbital and spin angular momenta
do repel each other differently (you showed him the plots in
Appendix~\ref{app:A}). But still, he won't
recognize which configuration has the highest energy, because this is
not at all a trivial problem! If you are also curious for the answer, I put
it here directly: the state
\begin{equation*}
\frac{1}{\sqrt{3}}
\begin{array}{|c|c|c|}
\hline
 & \phantom{\bullet} & \bullet \\ \hline
\bullet &  &  \\
\hline
\end{array}
- \frac{1}{\sqrt{3}}
\begin{array}{|c|c|c|}
\hline
\phantom{\bullet} & \bullet & \phantom{\bullet} \\ \hline
 & \bullet &  \\
\hline
\end{array}
+ \frac{1}{\sqrt{3}}
\begin{array}{|c|c|c|}
\hline
\bullet & \phantom{\bullet} & \\ \hline
 &  & \bullet \\
\hline
\end{array}
\end{equation*}
is an eigen-state of the $p^2$ configuration with the highest eigen-energy,
and it corresponds to the $^1S$ multiplet
(you see, nobody could answer this easily).
The complete problem is worked out step by step in Chapter~\ref{ch:5}.
Finally, in addition to Coulomb repulsion, we also included
spin-orbit coupling in Chapter~\ref{ch:6}, where we see the multiplet
spectral lines further split into finer structures.
Our work can be extended by introducing the $jj$-coupling,
where we consider each electron's total angular momentum. We can also
introduce external crystal field (our present work are in the frame of isolated
atoms) to study how our multiplet states respond to different external potential fields.

In the very end I must advertise our simulation tool: all the discussions in
this thesis have been implemented on a web page. Programming codes are written
in JavaScript. You can run simulations of all atoms on the periodic table
directly in a modern browser (no installation,
no compilation, and no plug-in).
This simulation tool can be accessed via the link:
\begin{center}
\url{www.cond-mat.de/sims/multiplet}
\end{center}

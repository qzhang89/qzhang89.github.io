\chapter{Spin-orbit coupling} \label{ch:6}

\section{The spin-orbit interaction}
In the very beginning of our discussion, we claimed that our problem is to solve the
Schr\"{o}dinger equation with the Hamiltonian
\begin{equation} \label{eq:Hamitt}
H = \sum_{i=1}^N \left[ -\frac{1}{2} \nabla_i^2 - \frac{Z}{r_i} \right] + \sum_{i<j}^N \frac{1}{|\vec{r}_i - \vec{r}_j|}
\end{equation}
We see clearly that each electron experiences a Coulomb
attraction from the nucleus and repulsions from all other electrons. However, this is
not quite the complete story. In fact, additionally, each electron also experiences a (weak)
magnetic force. Do you see where this magnetic force comes from?

Imagine you ``sit'' on an electron. From your point of view, the positively charged
nucleus is circling around you. This moving charge creates a current loop which
generates a magnetic field $\vec{B}$ (can be calculated from the Biot-Savart law).
On the other hand, the spinning electron has a magnetic dipole moment $\boldsymbol{\mu}_e$
which experiences the force from the magnetic field
\begin{equation} \label{eq:Bmu}
\vec{F} = \boldsymbol{\nabla}(\boldsymbol{\mu}_e\cdot\vec{B})
\end{equation}
The corresponding potential energy of the electron under this magnetic field is
\begin{equation} \label{eq:SOBmu}
H_\text{SO} = -\boldsymbol{\mu}_e\cdot\vec{B}
\end{equation}
%
It would be more convenient if we express the magnetic dipole moment in terms
of the spin angular momentum $\vec{S}$ and the magnetic field in terms of the
orbit angular momentum $\vec{L}$ \cite{WB},
\begin{equation} \label{eq:SOLS}
H_\text{SO} = \xi(r) \vec{L}\cdot\vec{S}
\end{equation}
where,\footnote{Some authors include an ``$\hbar^2$'' in the expression of $\xi(r)$.
But we prefer to absorb this $\hbar^2$ into the angular momenta, since $\vec{L}=
\vec{r}\times\vec{p}=\vec{r}\times (-i\hbar)\boldsymbol{\nabla}$ and $\vec{L}\cdot\vec{S}$
includes the $\hbar^2$ implicitly. Notice that in atomic units $\hbar=1$.}
\begin{equation} \label{eq:xir}
\xi(r) = \frac{1}{2m_e^2c^2} \frac{1}{r} \frac{dV}{dr}
\end{equation}
and $V(r)$ is our old friend, the (spherically symmetric) electric potential.
For hydrogen-like atoms,
\begin{equation}
V(r) = -\frac{1}{4\pi\epsilon_0}\frac{Ze^2}{r}
\quad \text{and} \quad
\xi(r) = \frac{1}{2m_e^2c^2} \frac{1}{4\pi\epsilon_0}\frac{Ze^2}{r^3}
\end{equation}
which agrees with the hydrogen atom spin-orbit interaction
discussed in Griffiths' book \cite{QM}. Now consider
an $N$-electron system, Eqn.~(\ref{eq:SOLS}) becomes
\begin{equation} \label{eq:HSO}
\boxed{
H_\text{SO} = \sum_{i=1}^N \xi(r_i) \boldsymbol{\ell}_i\cdot\vec{s}_i
}
\end{equation}
We reserved the capital letters for the total angular momentum operators
\begin{equation}
\vec{L} = \sum_{i=1}^N \boldsymbol{\ell}_i
\quad \text{and} \quad
\vec{S} = \sum_{i=1}^N \vec{s}_i
\end{equation}
This is the additional spin-orbit interaction Hamiltonian. It is rather a tiny
perturbation. In Eqn.~(\ref{eq:xir}), we see a ``speed of light squared''
factor in the denominator. Hence, spin-orbit is in general a weak interaction.
We introduced the atomic units (a.u.) in the beginning and this is the first time
that we encounter the speed of light. We know that in SI units, the speed of light is
\begin{equation} \label{eq:cSI}
c = 2.99792458\times10^8\;\mathrm{m}/\mathrm{s}
\end{equation}
%
To convert the speed of light from SI to atomic unit, we need the length and time
in atomic units:
\begin{align}
1\;\mathrm{a_0} & \approx 5.2918\times10^{-11}\;\mathrm{m}  \\
1\;\mathrm{t_0} & \approx 2.4189\times10^{-17}\;\mathrm{s}
\end{align}
%
We can convert easily, in a.u.\footnote{The inverse of this number is called the
fine structure constant $\alpha\equiv\frac{e^2}{4\pi\epsilon_0\hbar c}\approx 1/137.036$,
which is a dimensionless quantity. By the way, it is such a profound number that
all good theoretical physicists put this number up on their wall and worry about it,
said Mr.\ Feynman.}
\begin{equation} \label{eq:cau}
c \approx 137.036\;\mathrm{a_0}/\mathrm{t_0}
\end{equation}
%
Now we append $H_\text{SO}$ to Hamiltonian (\ref{eq:Hamitt}),
\begin{equation} \label{eq:HamitSO}
H = \sum_{i=1}^N \left[ -\frac{1}{2} \nabla_i^2 - \frac{Z}{r_i} \right] + \sum_{i<j}^N \frac{1}{|\vec{r}_i - \vec{r}_j|} + \sum_{i=1}^N \xi(r_i) \boldsymbol{\ell}_i\cdot\vec{s}_i
\end{equation}
which is the full Hamiltonian with spin-orbit coupling correction.

\section{Spin-orbit coupling within multiplet terms}
If there were no spin-orbit coupling, our Hamiltonian (\ref{eq:Hamitt}) commuted
with operators $\vec{L}$ and $\vec{S}$, which indicates that these
quantities are conserved \cite{QM}. That is why we could label our eigen-vectors as
\begin{equation} \label{eq:LMLSMS}
\Ket{L,M_L,S,M_S}
\end{equation}
%
Now, in the presence of spin-orbit coupling, our Hamiltonian (\ref{eq:HamitSO}) no
longer commutes with $\vec{L}$ and $\vec{S}$. The spin-orbit interaction mixed them
up. However, the Hamiltonian still commutes with the total angular momentum
\begin{equation} \label{eq:JLS}
\vec{J} = \vec{L} + \vec{S}
\end{equation}
%
If we only consider the spin-orbit interaction
within multiplet terms (for the same $L$ and $S$),
we can represent our eigen-vectors by\footnote{Notice that $\Ket{L,S,J,M_J}$ is
in general not an eigen-vector of the full Hamiltonian since the spin-orbit
interaction mixes different $L$ and $S$. But if we assume the spin-orbit splitting
$\ll$ the multiplet splitting, we can approximately take $\Ket{L,S,J,M_J}$ as our
eigen-vector (first order perturbation theory).}
\begin{equation} \label{eq:LSJMJ}
\Ket{L,S,J,M_J}
\end{equation}
which is a Clebsch-Grodan basis transformation \cite{WB} from basis (\ref{eq:LMLSMS}).
The possible values of $J$ are $L+S,L+S-1,\ldots,|L-S|$. For instance, for multiplet
$^3P$ with $L=1$ and $S=1$, we have $J=2,1,0$. This splits the degenerate $^3P$ into
$^3P_2,{^3P_1},{^3P_0}$. We now denote a term symbol as
\begin{equation} \label{eq:SLJ}
^{2S+1}L_J
\end{equation}
with
\begin{equation} \label{eq:SLJDegen}
\mathrm{degeneracy} = 2J+1
\end{equation}
which is less degenerate than a $^{2S+1}L$ term that we used previously.
The sum over all $(2J+1)$ within the same multiplet is equal to $(2S+1)(2L+1)$.
Now, our task is to obtain the eigen-energies of the $^{2S+1}L_J$ terms. Notice that,
\begin{equation} \label{eq:Jsq}
J^2 = L^2 + 2\vec{L}\cdot\vec{S} + S^2
\end{equation}
%
Hence, we can evaluate the following matrix element easily,
\begin{align} \label{eq:LSelem}
\Bra{LSJM_J}\vec{L}\cdot\vec{S}\Ket{LSJM_J} & = \frac{1}{2}\Bra{LSJM_J}J^2-L^2-S^2\Ket{LSJM_J} \nonumber \\
& = \frac{1}{2}[J(J+1)-L(L+1)-S(S+1)]
\end{align}
%
But this is not really what we are looking for. The spin-orbit eigen-energies
within multiplet terms should be the matrix element
\begin{equation} \label{eq:HSOelem}
\Bra{LSJM_J}H_\text{SO}\Ket{LSJM_J} = \Bra{LSJM_J} \sum_{i=1}^N \xi(r_i) \boldsymbol{\ell}_i\cdot\vec{s}_i \Ket{LSJM_J}
\end{equation}
which is proportional to the matrix element in Eqn.~(\ref{eq:LSelem}) (see Reference \cite{WB})
\begin{equation} \label{eq:HSOpropLS}
\Bra{LSJM_J} \sum_{i=1}^N \xi(r_i) \boldsymbol{\ell}_i\cdot\vec{s}_i \Ket{LSJM_J}
= A(nl,LS) \Bra{LSJM_J}\vec{L}\cdot\vec{S}\Ket{LSJM_J}
\end{equation}
%
The proportionality factor $A(nl,LS)$ depends on the radial wave function $u_{nl}$ and the
angular momenta $L$ and $S$. The eigen-energies lifted by the spin-orbit interaction are therefore,
\begin{equation} \label{eq:ESO}
\boxed{
E_\text{SO} = A(nl,LS) \frac{1}{2}[J(J+1)-L(L+1)-S(S+1)]
}
\end{equation}
%
Our problem is to find $A(nl,LS)$. Actually, we would
get the same proportionality factor if we considered the matrix elements
\begin{equation} \label{eq:HSOpropLS2}
\Bra{LM_LSM_S} \sum_{i=1}^N \xi(r_i) \boldsymbol{\ell}_i\cdot\vec{s}_i \Ket{LM_LSM_S}
= A(nl,LS) \Bra{LM_LSM_S}\vec{L}\cdot\vec{S}\Ket{LM_LSM_S}
\end{equation}
from which we are able to derive the expression for $A(nl,LS)$.
Let's first consider
\begin{equation*}
\Bra{LM_LSM_S}\vec{L}\cdot\vec{S}\Ket{LM_LSM_S}
\end{equation*}
To evaluate this matrix element, we expand the dot product
\begin{equation} \label{eq:LSdot}
\vec{L}\cdot\vec{S} = L_xS_x + L_yS_y + L_zS_z
\end{equation}
Express the $x$ and $y$ components in terms of ladder operators
\begin{equation} \label{eq:LxyLadder}
L_x = \frac{L_++L_-}{2} \quad \text{and} \quad L_y = \frac{L_+-L_-}{2i}
\end{equation}
We obtain,
\begin{align} \label{eq:ArightExp}
\Bra{LM_LSM_S}\vec{L}\cdot\vec{S}\Ket{LM_LSM_S} &
= \Bra{LM_LSM_S} \frac{1}{2}L_+S_- + \frac{1}{2}L_-S_+ + L_zS_z \Ket{LM_LSM_S} \nonumber \\
& = \Bra{LM_LSM_S} L_zS_z \Ket{LM_LSM_S} \nonumber \\
& = M_LM_S
\end{align}
%
Done! Keep in mind that our task is to find $A(nl,LS)$ in Eqn.~(\ref{eq:HSOpropLS2}).
Now, we consider
\begin{equation*}
\Bra{LM_LSM_S} \sum_{i=1}^N \xi(r_i) \boldsymbol{\ell}_i\cdot\vec{s}_i \Ket{LM_LSM_S}
\end{equation*}
which can be split into two independent parts
\begin{equation*}
\Bra{nl}\xi(r)\Ket{nl} \Bra{LM_LSM_S} \sum_{i=1}^N \boldsymbol{\ell}_i\cdot\vec{s}_i \Ket{LM_LSM_S}
\end{equation*}
The expectation vale $\Bra{nl}\xi(r)\Ket{nl}$ can be calculated easily by
numerical integration methods. But we encounter some difficulties with the second part,
where the single electron operators act on the eigen-state
which is in general a linear combination of configuration basis vectors.
For example, in $p^3$ configuration
we have,
\begin{equation}
|1,1,\frac{1}{2},\frac{1}{2}\rangle = \frac{1}{\sqrt{2}} \left( c_{1\downarrow}^\dagger c_{-1\uparrow}^\dagger c_{1\uparrow}^\dagger -c_{0\downarrow}^\dagger c_{0\uparrow}^\dagger c_{1\uparrow}^\dagger \right)|0\rangle
\end{equation}
%
It would be convenient if we expand the eigen-vectors in terms of configuration
basis vectors.
\begin{align}
\Ket{LM_LSM_S} & = \sum_{n=1}^\text{dim} a_n\, c_{m_N\sigma_N}^\dagger \ldots c_{m_2\sigma_2}^\dagger c_{m_1\sigma_1}^\dagger \Ket{0} \nonumber \\
& = \sum_{n=1}^\text{dim} a_n \Ket{L\{m_i\}S\{\sigma_i\}}
\end{align}
%
Hence,
\vspace{-0.1em}
\begin{align} \label{eq:mssum}
{} & \Bra{LM_LSM_S} \sum_{i=1}^N \boldsymbol{\ell}_i\cdot\vec{s}_i \Ket{LM_LSM_S} \nonumber \\
= {} & \left(\sum_{n=1}^\text{dim} \conj{a_n} \Bra{L\{m_i\}S\{\sigma_i\}}\right) \sum_{i=1}^N \boldsymbol{\ell}_i\cdot\vec{s}_i \left(\sum_{n=1}^\text{dim} a_n \Ket{L\{m_i\}S\{\sigma_i\}}\right) \nonumber \\
= {} & \sum_{n=1}^\text{dim} \left( |a_n|^2 \Bra{L\{m_i\}S\{\sigma_i\}} \sum_{i=1}^N \boldsymbol{\ell}_i\cdot\vec{s}_i \Ket{L\{m_i\}S\{\sigma_i\}} \right) \nonumber \\
= {} & \sum_{n=1}^\text{dim} \left( |a_n|^2 \sum_{i=1}^N \Bra{Lm_iS\sigma_i} \boldsymbol{\ell}_i\cdot\vec{s}_i \Ket{Lm_iS\sigma_i} \right) \nonumber \\
= {} & \sum_{n=1}^\text{dim} \left( |a_n|^2 \sum_{i=1}^N \Bra{Lm_iS\sigma_i} \frac{1}{2}\ell_+^is_-^i + \frac{1}{2}\ell_-^is_+^i + \ell_z^is_z^i \Ket{Lm_iS\sigma_i} \right) \nonumber \\
= {} & \sum_{n=1}^\text{dim} \left( |a_n|^2 \sum_{i=1}^N \Bra{Lm_iS\sigma_i} \ell_z^is_z^i \Ket{Lm_iS\sigma_i} \right) \nonumber \\
= {} & \sum_{n=1}^\text{dim} \left( |a_n|^2 \sum_{i=1}^N m_i \sigma_i \right)
\end{align}
%
Putting everything into Eqn.~(\ref{eq:HSOpropLS2}), we find
\begin{equation} \label{eq:AnlLS}
\boxed{
A(nl,LS) = \Bra{nl}\xi(r)\Ket{nl} \frac{\sum_{n=1}^\text{dim}\left( \left|a_n\right|^2 \sum_{i=1}^N m_i \sigma_i \right)}{M_LM_S}
}
\end{equation}
%
$A(nl,LS)$ has no dependence on $M_L$ or $M_S$. The most convenient choice would be
using the maximum values $M_L=L$ and $M_S=S$. One might worry about the denominator
since $L$ or $S$ might be zero. However, either $L$ or $S$ is zero indicates that
there is no spin-orbit interaction ($J=L+S=|L-S|$), hence no splitting, no worry.

I understand that the expression of $A(nl,LS)$ looks a bit complicated.
It would be nice to have a few worked examples. For convenience,
we break Eqn.~(\ref{eq:AnlLS}) into two parts,
\begin{align}
X(nl) & = \Bra{nl}\xi(r)\Ket{nl} \\
M(LS) & = \frac{\sum_{n=1}^\text{dim}\left( \left|a_n\right|^2 \sum_{i=1}^N m_i \sigma_i \right)}{M_LM_S}
\end{align}

\paragraph{Example 1:} $^3P$ multiplet in $p^2$ configuration.

The leading vector
\begin{equation*}
|1,1,1,1\rangle = c_{0\uparrow}^\dagger c_{1\uparrow}^\dagger |0\rangle
\end{equation*}
\begin{equation*}
M({^3P}) = \frac{0\times \frac{1}{2} + 1\times \frac{1}{2}}{1\times 1} = \frac{1}{2}
\end{equation*}
Hence,
\begin{equation*}
A(np, {^3P}) = \frac{1}{2} X(np)
\end{equation*}

\paragraph{Example 2:} $^2G$ multiplet in $d^3$ configuration.

The leading vector
\begin{equation*}
|4,4,\frac{1}{2},\frac{1}{2}\rangle = \frac{1}{\sqrt{5}} \left( \sqrt{2}c_{2\downarrow}^\dagger c_{0\uparrow}^\dagger c_{2\uparrow}^\dagger -\sqrt{3}c_{1\downarrow}^\dagger c_{1\uparrow}^\dagger c_{2\uparrow}^\dagger \right)|0\rangle
\end{equation*}
\begin{equation*}
M({^2G})
= \frac{\frac{2}{5}(2\times -\frac{1}{2} + 0\times \frac{1}{2} + 2\times \frac{1}{2}) +
\frac{3}{5}(1\times -\frac{1}{2} + 1\times \frac{1}{2} + 2\times \frac{1}{2})
}{4\times \frac{1}{2}} = \frac{3}{10}
\end{equation*}
Hence,
\begin{equation*}
A(nd, {^2G}) = \frac{3}{10} X(nd)
\end{equation*}

There is a symmetry property of the factor $M(LS)$. If the shell is less than half-filled,
$M(LS)>0$; if the shell is more than half-filled, $M(LS)<0$; and if the shell is
exactly half-filled, $M(LS)=0$ (no splitting).

The remaining part $X(nl)$, is relatively more straightforward.
\begin{align} \label{eq:Xnlint}
X(nl) & = \Bra{nl}\xi(r)\Ket{nl} = \int_0^\infty dr\,r^2\, \conj{R_{nl}}(r) \xi(r) R_{nl}(r) \nonumber \\
& = \int_0^\infty dr\, \conj{u_{nl}}(r) \xi(r) u_{nl}(r) = \int_0^\infty dr\, |u_{nl}(r)|^2 \xi(r)
\end{align}
and in atomic unit,
\begin{equation}
\xi(r) = \frac{1}{2\times 137.036^2} \frac{1}{r} \frac{dV}{dr}
\end{equation}
%
In our self-consistent field approximation, the potential $V(r)$ is the mean-field
electric potential that all electrons experience in.
The first order derivative can be obtained approximately from a finite
difference formula
\begin{equation}
\frac{dV}{dx} \approx \frac{V(x+\Delta x)-V(x-\Delta x)}{2\Delta x}
\end{equation}
Remember the logarithmic grid transformation in Eqn~(\ref{eq:x2r}),
we have,
\begin{equation}
dr=rdx \quad \text{and} \quad \frac{dV}{dr} = \frac{1}{r} \frac{dV}{dx}
\end{equation}
%
Hence, numerically,
\begin{equation} \label{eq:Xnl}
X(nl) = \frac{1}{2\times 137.036^2} \int_{x_\text{min}}^{x_\text{max}} dx\, |u_{nl}(x)|^2 \frac{1}{r} \frac{dV}{dx}
\end{equation}

To get an intuitive understanding of this spin-orbit energy splitting,
we again take the carbon atom as an example. A carbon atom has electronic
configuration $1s^2$ $2s^2$ $2p^2$. From the self-consistent calculation,
we obtained both the mean-field potential $V(r)$ and the wave function
$u_{2p}(r)$. From Eqn.~(\ref{eq:Xnl}), we can calculate
\begin{equation} \label{eq:Xcarbon}
X(2p) = 0.000218\ \text{(Hartree)}
\end{equation}
%
Meanwhile, we have discovered that a $p^2$ configuration produces
$^3P$, $^1D$ and $^1S$ multiplets. Among these three multiplets,
only $^3P$ splits due to spin-orbit interaction. Both $^1D$ and $^1S$
do not split since $S=0$. In the previous examples, we found
for the $p^2$ configuration,
\begin{equation}
M(^3P) = \frac{1}{2}
\end{equation}
%
Therefore,
\begin{equation}
A(2p^2,{^3P}) = 0.000218\times\frac{1}{2} = 0.000109\ \text{(Hartree)}
\end{equation}
Now the spin-orbit energy splittings within $^3P$ are (Eqn.~(\ref{eq:ESO}))
\begin{align}
\begin{split}
E_\text{SO}(^3P_2) & = 0.000109\frac{1}{2}[2(2+1)-1(1+1)-1(1+1)] = \phantom{-}0.000109\ (\text{Hartree}) \\
E_\text{SO}(^3P_1) & = 0.000109\frac{1}{2}[1(1+1)-1(1+1)-1(1+1)] = -0.000109\ (\text{Hartree}) \\
E_\text{SO}(^3P_0) & = 0.000109\frac{1}{2}[0(0+1)-1(1+1)-1(1+1)] = -0.000218\ (\text{Hartree})
\end{split}
\end{align}
with this additional splitting, Fig.~\ref{fig:p2split} gets an extra column.
\begin{figure}[h!]
\begin{center}
\begin{tikzpicture}[scale=0.175]
\newcommand{\LA}{0}
\newcommand{\RA}{15}
\newcommand{\LB}{30}
\newcommand{\RB}{45}
\newcommand{\LC}{60}
\newcommand{\RC}{75}
\newcommand{\EA}{13.7797}
\newcommand{\EBa}{27.5594}
\newcommand{\EBb}{11.0236}
\newcommand{\EBc}{0}
\newcommand{\ECa}{27.5594}
\newcommand{\ECb}{11.0236}
\newcommand{\ECc}{ 4.36}
\newcommand{\ECd}{-4.36}
\newcommand{\ECe}{-8.72}
%
\draw[very thick] (\LA,\EA) -- (\RA,\EA);
%
\draw[very thick] (\LB,\EBa) -- (\RB,\EBa);
\draw[very thick] (\LB,\EBb) -- (\RB,\EBb);
\draw[very thick] (\LB,\EBc) -- (\RB,\EBc);
%
\draw[very thick] (\LC,\ECa) -- (\RC,\ECa);
\draw[very thick] (\LC,\ECb) -- (\RC,\ECb);
\draw[very thick] (\LC,\ECc) -- (\RC,\ECc);
\draw[very thick] (\LC,\ECd) -- (\RC,\ECd);
\draw[very thick] (\LC,\ECe) -- (\RC,\ECe);
%
\node at (\LA+8,\EA+2) {$p^2$};
\node at (\LB+7,\EBa+2) {$^1S$};
\node at (\LB+7,\EBb+2) {$^1D$};
\node at (\LB+7,\EBc+2) {$^3P$};
\node at (\LC+7,\ECa+2) {$^1S_0$};
\node at (\LC+7,\ECb+2) {$^1D_2$};
\node at (\LC+7,\ECc+2) {$^3P_2$};
\node at (\LC+7,\ECd+2) {$^3P_1$};
\node at (\LC+7,\ECe+2) {$^3P_0$};
%
\draw[very thin, gray, dashed] (\RA,\EA) -- (\LB,\EBa);
\draw[very thin, gray, dashed] (\RA,\EA) -- (\LB,\EBb);
\draw[very thin, gray, dashed] (\RA,\EA) -- (\LB,\EBc);
%
\draw[very thin, gray, dashed] (\RB,\EBa) -- (\LC,\ECa);
\draw[very thin, gray, dashed] (\RB,\EBb) -- (\LC,\ECb);
\draw[very thin, gray, dashed] (\RB,\EBc) -- (\LC,\ECc);
\draw[very thin, gray, dashed] (\RB,\EBc) -- (\LC,\ECd);
\draw[very thin, gray, dashed] (\RB,\EBc) -- (\LC,\ECe);
%
\draw[triangle 45-triangle 45] (\LB+12,\EBa) -- (\LB+12,\EBb);
\draw[triangle 45-triangle 45] (\LB+12,\EBa) -- (\LB+12,\EBc);
\draw[triangle 45-triangle 45] (\LC+12,\ECc) -- (\LC+12,\ECd);
\draw[triangle 45-triangle 45] (\LC+12,\ECd) -- (\LC+12,\ECe);
%
\node at (\RB+5,19.2915) {$\Delta E = 0.082679$};
\node at (\RB+5,5.5118) {$\Delta E = 0.055118$};
\node at (\RC+5,0) {$\Delta E = 0.000218$};
\node at (\RC+5,-6.54) {$\Delta E = 0.000109$};
%
\node at (\LA+7.5,-14) {Configuration};
\node at (\LB+7.5,-14) {Coulomb repulsion};
\node at (\LC+7.5,-14) {Spin-orbit};
\end{tikzpicture}
\end{center}
\caption{Energy splitting (Coulomb repulsion plus spin-orbit interaction)
of the $p^2$ configuration of a carbon atom. Energies are given in units of Hartree (a.u.).
The spin-orbit splitting is magnified for a better plotting. (see Fig.~\ref{fig:cvspb}
for realistic scale.)}
\label{fig:p2splitSO}
\end{figure}

The spin-orbit splitting in Fig.~\ref{fig:p2splitSO} is magnified for a better plotting.
The actual splitting is rather tiny (remember the $1/c^2$ factor). But I want to
emphasize that there are two issues about the spin-orbit splitting, namely,
the ``scale'' and the ``shape''. This can be seen from Eqn.~(\ref{eq:ESO}).
The part
\begin{equation*}
[J(J+1)-L(L+1)-S(S+1)]
\end{equation*}
leads to the splitting ``shape'', which is determined by the multiplet term $^{2S+1}L$.
While the other part
\begin{equation*}
A(nl,LS)
\end{equation*}
controls the ``scale'' of the splitting.
We have seen that $A(nl,LS)$ can be separated into $X(nl)$ and $M(LS)$. For a
given multiplet term, $M(LS)$ is fixed. The final term that governs the scale is
\begin{equation*}
X(nl) = \Bra{nl}\xi(r)\Ket{nl} 
\end{equation*}
which is the coupling strength constant. For a not-so-heavy atom, $X(nl)$ is usually
weak (see Eqn.(\ref{eq:Xcarbon}) for a carbon atom). We can safely simplify
the spin-orbit interaction within a multiplet term and obtain surprisingly
accurate energies. But for a heavy atom, like uranium,
this spin-orbit coupling constant becomes strong due to the deep radial potential.
If the order of the spin-orbit splitting reaches the order of the multiplet
energy splitting, our ``spin-orbit within multiplet terms'' may no longer be
a good approximation. In this case, we should consider the spin-orbit interactions
within the entire shell and diagonalize the complete Hamiltonian.

\section{Spin-orbit coupling within the entire shell}
By saying ``within entire shell'', it means to solve the problem in the complete basis
of a given configuration, say, $p^2$,
\begin{center}
\vspace{-0.5em}
\begin{tabular}{|c|c|c|}
\hline
$\bullet$ & $\bullet$ & $\phantom{\bullet}$ \\ \hline
 &  &  \\
\hline
\end{tabular}
\begin{tabular}{|c|c|c|}
\hline
$\bullet$ & $\phantom{\bullet}$ & $\bullet$ \\ \hline
 &  &  \\
\hline
\end{tabular}
\begin{tabular}{|c|c|c|}
\hline
$\bullet$ & $\phantom{\bullet}$ & $\phantom{\bullet}$ \\ \hline
$\bullet$ &  &  \\
\hline
\end{tabular}
\begin{tabular}{|c|c|c|}
\hline
$\bullet$ & $\phantom{\bullet}$ & $\phantom{\bullet}$ \\ \hline
 & $\bullet$ &  \\
\hline
\end{tabular}
\begin{tabular}{|c|c|c|}
\hline
$\bullet$ & $\phantom{\bullet}$ & $\phantom{\bullet}$ \\ \hline
 &  & $\bullet$ \\
\hline
\end{tabular} \\
\vspace{0.5em}
\begin{tabular}{|c|c|c|}
\hline
$\phantom{\bullet}$ & $\bullet$ & $\bullet$ \\ \hline
 &  &  \\
\hline
\end{tabular}
\begin{tabular}{|c|c|c|}
\hline
$\phantom{\bullet}$ & $\bullet$ & $\phantom{\bullet}$ \\ \hline
$\bullet$ &  &  \\
\hline
\end{tabular}
\begin{tabular}{|c|c|c|}
\hline
$\phantom{\bullet}$ & $\bullet$ & $\phantom{\bullet}$ \\ \hline
 & $\bullet$ &  \\
\hline
\end{tabular}
\begin{tabular}{|c|c|c|}
\hline
$\phantom{\bullet}$ & $\bullet$ & $\phantom{\bullet}$ \\ \hline
 &  & $\bullet$ \\
\hline
\end{tabular}
\begin{tabular}{|c|c|c|}
\hline
$\phantom{\bullet}$ & $\phantom{\bullet}$ & $\bullet$ \\ \hline
$\bullet$ &  &  \\
\hline
\end{tabular} \\
\vspace{0.5em}
\begin{tabular}{|c|c|c|}
\hline
$\phantom{\bullet}$ & $\phantom{\bullet}$ & $\bullet$ \\ \hline
 & $\bullet$ &  \\
\hline
\end{tabular}
\begin{tabular}{|c|c|c|}
\hline
$\phantom{\bullet}$ & $\phantom{\bullet}$ & $\bullet$ \\ \hline
 &  & $\bullet$ \\
\hline
\end{tabular}
\begin{tabular}{|c|c|c|}
\hline
$\phantom{\bullet}$ & $\phantom{\bullet}$ & $\phantom{\bullet}$ \\ \hline
$\bullet$ & $\bullet$ &  \\
\hline
\end{tabular}
\begin{tabular}{|c|c|c|}
\hline
$\phantom{\bullet}$ & $\phantom{\bullet}$ & $\phantom{\bullet}$ \\ \hline
$\bullet$ &  & $\bullet$ \\
\hline
\end{tabular}
\begin{tabular}{|c|c|c|}
\hline
$\phantom{\bullet}$ & $\phantom{\bullet}$ & $\phantom{\bullet}$ \\ \hline
 & $\bullet$ & $\bullet$ \\
\hline
\end{tabular}
\end{center}
%
We have previously used the same basis when solving the Coulomb repulsion problem.
Since we have the basis already, the remaining task is to set up the matrix
representation of the spin-orbit Hamiltonian in our basis.
To construct the matrix representation, the first step is to reformulate the Hamiltonian
\begin{equation}
H_\text{SO} = \sum_{i=1}^N \xi(r_i) \boldsymbol{\ell}_i\cdot\vec{s}_i
\end{equation}
into second quantization,
\begin{equation} \label{eq:SO2nd}
H_\text{SO} = \sum_{\alpha,\beta} V_{\alpha\beta} c_\alpha^\dagger c_\beta
\end{equation}
where,
\begin{align*}
\alpha & = \{n_1,\,l_1,\,m_1,\,\sigma_1\} \\
\beta  & = \{n_2,\,l_2,\,m_2,\,\sigma_2\}
\end{align*}
enumerate all possible quantum states of electrons. Here we consider only
interactions within the same shell, so we restrict $n_1l_1=n_2l_2=nl$.
If you still remember, we devoted an entire chapter
calculating the Coulomb repulsion matrix element $U_{\alpha\beta\gamma\delta}$
since it was extremely complicated.
However, today, our spin-orbit matrix element
\begin{equation}
V_{\alpha\beta} = \Bra{\alpha} \xi(r) \boldsymbol{\ell}\cdot\vec{s} \Ket{\beta}
\end{equation}
can be calculated with zero difficulty.
This spin-orbit matrix element can be split into a radial dependent part
and an angular dependent part
\begin{equation}
V_{\alpha\beta} = \Bra{nl} \xi(r) \Ket{nl} \Bra{m_1\sigma_1} \boldsymbol{\ell}\cdot\vec{s} \Ket{m_2\sigma_2}
\end{equation}
%
The radial part $\Bra{nl}\xi(r)\Ket{nl}$ is identical to $X(nl)$
which we have discussed in Eqn.~(\ref{eq:Xnlint}). And the angular part,
\begin{align} \label{eq:AngularSO}
\Bra{m_1\sigma_1} \boldsymbol{\ell}\cdot\vec{s} \Ket{m_2\sigma_2}
& = \Bra{m_1\sigma_1} \ell_x s_x + \ell_y s_y + \ell_z s_z \Ket{m_2\sigma_2} \nonumber \\
& = \Bra{m_1\sigma_1} \frac{1}{2}\ell_+ s_- + \frac{1}{2}\ell_- s_+ + \ell_z s_z \Ket{m_2\sigma_2} \nonumber \\
& = \phantom{+} \frac{1}{2}\sqrt{(l+m_2+1)(l-m_2)\left(\frac{1}{2}+\sigma_2\right)\left(\frac{1}{2}-\sigma_2+1\right)} \Braket{m_1\sigma_1|m_2+1,\sigma_2-1} \nonumber \\
& \phantom{=}\, + \frac{1}{2}\sqrt{(l+m_2)(l-m_2+1)\left(\frac{1}{2}+\sigma_2+1\right)\left(\frac{1}{2}-\sigma_2\right)} \Braket{m_1\sigma_1|m_2-1,\sigma_2+1} \nonumber \\
& \phantom{=}\, + m_2\sigma_2\Braket{m_1\sigma_1|m_2\sigma_2}
\end{align}
can be computed easily with the orthonormality of angular wave functions,
\begin{equation}
\Braket{m_1\sigma_1|m_2\sigma_2} = \delta_{m_1m_2}\delta_{\sigma_1\sigma_2}
\end{equation}

Setting up the spin-orbit Hamiltonian is simpler than setting up the Coulomb
repulsion Hamiltonian since we have only $\alpha$ and $\beta$ indices
\begin{equation} \label{eq:Hsoij}
\Bra{i} H_\text{SO} \Ket{j} = 
\Bra{i} \sum_{\alpha,\beta} V_{\alpha\beta} c_\alpha^\dag c_\beta \Ket{j}
\end{equation}
%
Hence, the algorithm is also simpler with less for loops (comparing with Algorithm~\ref{alg:Huij}).
\begin{algorithm}[h!]
\caption{Set up Hamiltonian}\label{alg:Hsoij}
\begin{algorithmic}[1]
\Function{Hamiltonian}{$basis$}
\State $dim \gets basis.dim$
\For {$i \gets 0$ to $dim$}
\State $conf_i \gets basis.conf[i]$
\ForAll {$\alpha$}
\If {\Call{isBit}{$\alpha$, $conf_i$}}
\State $conf_\alpha \gets$ \Call{clearBit}{$\alpha$, $conf_i$}
\ForAll {$\beta$}
\If {\Call{!isBit}{$\beta$, $conf_\alpha$}}
\State $conf_\beta \gets$ \Call{setBit}{$\beta$, $conf_\alpha$}
\State $conf_j \gets conf_\beta$
\State $j \gets basis.index[conf_j]$
\State $H_\text{SO}[i,j] \gets H_\text{SO}[i,j] + fsign*V_{\alpha\beta}$
\EndIf
\EndFor
\EndIf
\EndFor
\EndFor
\State \Return $H_\text{SO}$
\EndFunction
\end{algorithmic}
\end{algorithm}

If we diagonalize $H_\text{SO}$ directly, we would obtain the eigen-energies
of the pure spin-orbit interaction. To include both Coulomb repulsion and spin-orbit coupling,
we should diagonalize (numerically) the sum $(H_U+H_\text{SO})$.
For not-so-heavy atoms, like carbon, the resulting eigen-energies are surprisingly close
to the eigen-energies we obtained from the ``within multiplet terms'' approximation, which
is pretty remarkable, since the solutions from two different approaches agree each other.
However, for heavy atoms, there is a large discrepancy between those two solutions.

For heavy atoms, the deep potential leads to large values in the derivative $dV/dr$.
Hence, the spin-orbit coupling constant $\Bra{nl}\xi(r)\Ket{nl}$ is large. For
strong spin-orbit interactions, the order of energy splitting can reach the order
of the Coulomb repulsion splitting.
In this case the approximation using spin-orbit coupling within multiplet terms
are no longer appropriate.

This can be clearly demonstrated by a comparison between a carbon (C) and a lead (Pb)
atom, which are from the same group with the same open shell configuration $p^2$.
Table~\ref{table:cvspb} tabulated the numerical energies of spin-orbit interactions
within multiplet terms and within entire shell.

\begin{table}[h!]
\caption{Comparison of open shell spin-orbit energies within multiplet
terms and within entire shell for a carbon atom and a lead atom.
Energies are given in units of Hartree (a.u.).}
\label{table:cvspb}
\begin{center}
\begin{tabular}{ c | c | c | c | c | c }
  \hline
 Elem & Orbital & \begin{tabular}[t]{@{}c@{}}Energy within\\multiplet terms\\($\times$degeneracy)\end{tabular} &
 \begin{tabular}[t]{@{}c@{}}Energy within\\entire shell\\($\times$degeneracy)\end{tabular} & Abs Error & Rel Error \\ \hline \hline
    C & $2p^2$  & 0.612081 ($\times 1$) & 0.612081 ($\times 1$) & 0.000000 & 0.000000 \\
      &         & 0.529402 ($\times 5$) & 0.529403 ($\times 5$) & 0.000001 & 0.000002 \\
      &         & 0.474393 ($\times 5$) & 0.474392 ($\times 5$) & 0.000001 & 0.000002 \\
      &         & 0.474175 ($\times 3$) & 0.474175 ($\times 3$) & 0.000000 & 0.000000 \\
      &         & 0.474066 ($\times 1$) & 0.474065 ($\times 1$) & 0.000001 & 0.000002 \\ \hline
   Pb & $6p^2$  & 0.323500 ($\times 1$) & 0.335963 ($\times 1$) & 0.012463 & 0.037096 \\
      &         & 0.272826 ($\times 5$) & 0.284981 ($\times 5$) & 0.012155 & 0.042652 \\
      &         & 0.252988 ($\times 5$) & 0.240833 ($\times 5$) & 0.012155 & 0.050471 \\
      &         & 0.225100 ($\times 3$) & 0.225100 ($\times 3$) & 0.000000 & 0.000000 \\
      &         & 0.211156 ($\times 1$) & 0.198693 ($\times 1$) & 0.012463 & 0.062725 \\
  \hline  
\end{tabular}
\end{center}
\end{table}

The discrepancy can be seen more easily from the spectrum plot in Fig.~\ref{fig:cvspb}.
The spin-orbit splitting within multiplet terms and within
the entire shell are plotted in the 3rd and 4th column of the plot, respectively.
It is difficult to resolve the spin-orbit splitting in the carbon atom plot, since the
energy differences are so tiny. But this tiny splitting gives a good approximation
when considering spin-orbit coupling within multiplet terms. On the other hand,
the amplitude of spin-orbit splitting in the lead atom reaches the amplitude of Coulomb
repulsion splitting. In this case, the energies from the ``within multiplet terms''
approximation do not match the (more accurate) full shell diagonalization.

\begin{figure}
\centering
\subfloat[][Carbon (C)]{
\begin{tikzpicture}[scale=0.175]
\newcommand{\LA}{0}
\newcommand{\RA}{12}
\newcommand{\LB}{23}
\newcommand{\RB}{35}
\newcommand{\LC}{46}
\newcommand{\RC}{58}
\newcommand{\LD}{69}
\newcommand{\RD}{81}
\newcommand{\EA}{13.8016*1.5}
\newcommand{\EBa}{27.6032*1.5}
\newcommand{\EBb}{11.0674*1.5}
\newcommand{\EBc}{0.0438*1.5}
\newcommand{\ECa}{27.6032*1.5}
\newcommand{\ECb}{11.0674*1.5}
\newcommand{\ECc}{0.0656*1.5}
\newcommand{\ECd}{0.0220*1.5}
\newcommand{\ECe}{0.0002*1.5}
\newcommand{\EDa}{27.6032*1.5}
\newcommand{\EDb}{11.0676*1.5}
\newcommand{\EDc}{0.0658*1.5}
\newcommand{\EDd}{0.0220*1.5}
\newcommand{\EDe}{0.0000*1.5}
%
\draw[very thick] (\LA,\EA) -- (\RA,\EA);
%
\draw[very thick] (\LB,\EBa) -- (\RB,\EBa);
\draw[very thick] (\LB,\EBb) -- (\RB,\EBb);
\draw[very thick] (\LB,\EBc) -- (\RB,\EBc);
%
\draw[very thick] (\LC,\ECa) -- (\RC,\ECa);
\draw[very thick] (\LC,\ECb) -- (\RC,\ECb);
\draw[very thick] (\LC,\ECc) -- (\RC,\ECc);
\draw[very thick] (\LC,\ECd) -- (\RC,\ECd);
\draw[very thick] (\LC,\ECe) -- (\RC,\ECe);
%
\draw[very thick] (\LD,\EDa) -- (\RD,\EDa);
\draw[very thick] (\LD,\EDb) -- (\RD,\EDb);
\draw[very thick] (\LD,\EDc) -- (\RD,\EDc);
\draw[very thick] (\LD,\EDd) -- (\RD,\EDd);
\draw[very thick] (\LD,\EDe) -- (\RD,\EDe);
%
\draw[very thin, gray, dashed] (\RA,\EA) -- (\LB,\EBa);
\draw[very thin, gray, dashed] (\RA,\EA) -- (\LB,\EBb);
\draw[very thin, gray, dashed] (\RA,\EA) -- (\LB,\EBc);
%
\draw[very thin, gray, dashed] (\RB,\EBa) -- (\LC,\ECa);
\draw[very thin, gray, dashed] (\RB,\EBb) -- (\LC,\ECb);
\draw[very thin, gray, dashed] (\RB,\EBc) -- (\LC,\ECc);
\draw[very thin, gray, dashed] (\RB,\EBc) -- (\LC,\ECd);
\draw[very thin, gray, dashed] (\RB,\EBc) -- (\LC,\ECe);
%
\node at (\LA+6,\EA+2) {$p^2$};
\node at (\LB+6,\EBa+2) {$^1S\ $ 0.612081};
\node at (\LB+6,\EBb+2) {$^1D\ $ 0.529402};
\node at (\LB+6,\EBc+2) {$^3P\ $ 0.474284};
\node at (\LC+6,\ECa+2) {$^1S_0\ $ 0.612081};
\node at (\LC+6,\ECb+2) {$^1D_2\ $ 0.529402};
\node at (\LC+6,\ECc+6) {$^3P_2\ $ 0.474393};
\node at (\LC+6,\ECc+4) {$^3P_1\ $ 0.474175};
\node at (\LC+6,\ECc+2) {$^3P_0\ $ 0.474066};
\node at (\LD+6,\EDa+2) {0.612081};
\node at (\LD+6,\EDb+2) {0.529403};
\node at (\LD+6,\EDc+6) {0.474392};
\node at (\LD+6,\EDc+4) {0.474175};
\node at (\LD+6,\EDc+2) {0.474065};
%
\node at (\LA+6,-5) {Configuration};
\node at (\LB+6,-5) {Coulomb repulsion};
\node at (\LC+6,-5) {Spin-orbit};
\node at (\LD+6,-5) {$H_U+H_\text{SO}$};
\end{tikzpicture}
\label{fig:c}} \\
\subfloat[][Lead (Pb)]{
\begin{tikzpicture}[scale=0.175]
\newcommand{\LA}{0}
\newcommand{\RA}{12}
\newcommand{\LB}{23}
\newcommand{\RB}{35}
\newcommand{\LC}{46}
\newcommand{\RC}{58}
\newcommand{\LD}{69}
\newcommand{\RD}{81}
\newcommand{\EA}{13.7270*1.5}
\newcommand{\EBa}{24.9614*1.5}
\newcommand{\EBb}{14.8266*1.5}
\newcommand{\EBc}{ 8.0702*1.5}
\newcommand{\ECa}{24.9614*1.5}
\newcommand{\ECb}{14.8266*1.5}
\newcommand{\ECc}{10.8590*1.5}
\newcommand{\ECd}{ 5.2814*1.5}
\newcommand{\ECe}{ 2.4926*1.5}
\newcommand{\EDa}{27.4540*1.5}
\newcommand{\EDb}{17.2576*1.5}
\newcommand{\EDc}{ 8.4280*1.5}
\newcommand{\EDd}{ 5.2814*1.5}
\newcommand{\EDe}{ 0.0000*1.5}
%
\draw[very thick] (\LA,\EA) -- (\RA,\EA);
%
\draw[very thick] (\LB,\EBa) -- (\RB,\EBa);
\draw[very thick] (\LB,\EBb) -- (\RB,\EBb);
\draw[very thick] (\LB,\EBc) -- (\RB,\EBc);
%
\draw[very thick] (\LC,\ECa) -- (\RC,\ECa);
\draw[very thick] (\LC,\ECb) -- (\RC,\ECb);
\draw[very thick] (\LC,\ECc) -- (\RC,\ECc);
\draw[very thick] (\LC,\ECd) -- (\RC,\ECd);
\draw[very thick] (\LC,\ECe) -- (\RC,\ECe);
%
\draw[very thick] (\LD,\EDa) -- (\RD,\EDa);
\draw[very thick] (\LD,\EDb) -- (\RD,\EDb);
\draw[very thick] (\LD,\EDc) -- (\RD,\EDc);
\draw[very thick] (\LD,\EDd) -- (\RD,\EDd);
\draw[very thick] (\LD,\EDe) -- (\RD,\EDe);
%
\draw[very thin, gray, dashed] (\RA,\EA) -- (\LB,\EBa);
\draw[very thin, gray, dashed] (\RA,\EA) -- (\LB,\EBb);
\draw[very thin, gray, dashed] (\RA,\EA) -- (\LB,\EBc);
%
\draw[very thin, gray, dashed] (\RB,\EBa) -- (\LC,\ECa);
\draw[very thin, gray, dashed] (\RB,\EBb) -- (\LC,\ECb);
\draw[very thin, gray, dashed] (\RB,\EBc) -- (\LC,\ECc);
\draw[very thin, gray, dashed] (\RB,\EBc) -- (\LC,\ECd);
\draw[very thin, gray, dashed] (\RB,\EBc) -- (\LC,\ECe);
%
\node at (\LA+6,\EA+2) {$p^2$};
\node at (\LB+6,\EBa+2) {$^1S\ $ 0.323500};
\node at (\LB+6,\EBb+2) {$^1D\ $ 0.272826};
\node at (\LB+6,\EBc+2) {$^3P\ $ 0.239044};
\node at (\LC+6,\ECa+2) {$^1S_0\ $ 0.323500};
\node at (\LC+6,\ECb+2) {$^1D_2\ $ 0.272826};
\node at (\LC+6,\ECc+2) {$^3P_2\ $ 0.252988};
\node at (\LC+6,\ECd+2) {$^3P_1\ $ 0.225100};
\node at (\LC+6,\ECe+2) {$^3P_0\ $ 0.211156};
\node at (\LD+6,\EDa+2) {0.335963};
\node at (\LD+6,\EDb+2) {0.284981};
\node at (\LD+6,\EDc+2) {0.240833};
\node at (\LD+6,\EDd+2) {0.225100};
\node at (\LD+6,\EDe+2) {0.198693};
%
\node at (\LA+6,-5) {Configuration};
\node at (\LB+6,-5) {Coulomb repulsion};
\node at (\LC+6,-5) {Spin-orbit};
\node at (\LD+6,-5) {$H_U+H_\text{SO}$};
\end{tikzpicture}
\label{fig:pb}}
\caption{Comparison of open shell energy splitting between a ``light'' atom carbon (C)
and a ``heavy'' atom lead (Pb). Energies are given in units of Hartree (a.u.).
From left to right, 1st column: electronic configuration;
2nd column: Coulomb repulsion energy splitting; 3rd column: spin-orbit interaction
within multiplet terms; 4th column: eigen-energies of the Hamiltonian $(H_U+H_\text{SO})$.
Spin-orbit interaction within multiplet terms are good approximations for ``light'' atoms
but not for ``heavy'' atoms.}
\label{fig:cvspb}
\end{figure}

In the 4th column in Fig.~\ref{fig:cvspb}, we labeled each energy level by their numerical
values. But we didn't put a label like $^{2S+1}L$ or $^{2S+1}L_J$. This is because
when we consider the eigen-energies from the sum $(H_U+H_\text{SO})$, the energy levels are mixed
with contributions from different angular momenta $L$ and $S$. We can no longer
label each energy level as purely $^{2S+1}L$ or $^{2S+1}L_J$. Nevertheless, from the plot,
we do see some strong correspondence between the multiplet terms and the energy levels
from $(H_U+H_\text{SO})$. This correspondence can be calculated from the overlap between
the eigen-vectors of multiplet terms and the eigen-vectors of $(H_U+H_\text{SO})$, which
is known as the character of eigen-vectors.

For a multiplet term $^{2S+1}L$ (with seniority number $W$ if necessary), we have eigen-vectors
$\Ket{L,M_L,S,M_S}$ with $M_L=L,\ldots,-L$ and $M_S=S,\ldots,-S$. Those vectors span
a ``small space'' of this specific multiplet term. Now, suppose we have an
eigen-vector $\Ket{v}$ of $(H_U+H_\text{SO})$ from our numerical diagonalization.
To check if this eigen-vector $\Ket{v}$ lives inside this ``small space'',
we compute the character
\begin{equation} \label{eq:char}
\boxed{
\lambda = \sum_{M_L,M_S} \big|\Braket{L,M_L,S,M_S|v}\big|^2
}
\end{equation}
%
If $\Ket{v}$ lives completely in the space spanned by $\Ket{L,M_L,S,M_S}$,
we shall get $\lambda=1$. On the contrary, if $\Ket{v}$ is completely off,
we will get $\lambda=0$. However, in our problem, this $\Ket{v}$ is often
partly in one multiplet term and partly in the others. In this case we get
$0<\lambda<1$. The closer to 1, the stronger is the contribution from
this specific multiplet term.

Continuing with our discussion, we compute all characters for each numerical
eigen-vectors of $(H_U+H_\text{SO})$ within the three multiplet spaces $^1S$, $^1D$
and $^3P$. The characters are listed in Table~\ref{table:char}.

\begin{table}[h!]
\caption{Character of numerical eigen-vectors in different multiplet term spaces.
We highlight the main contribution using underlines.
Zero values are left as empty entries so that the structure can be seen clearly.}
\label{table:char}
\begin{center}
\begin{tabular}{ c | c | c | c | c | c }
  \hline
 Elem & Orbital & \begin{tabular}[t]{@{}c@{}}Energy within\\entire shell\\($\times$degeneracy)\end{tabular} &
 \begin{tabular}[t]{@{}c@{}}Character\\in $^1S$\end{tabular} &
 \begin{tabular}[t]{@{}c@{}}Character\\in $^1D$\end{tabular} &
 \begin{tabular}[t]{@{}c@{}}Character\\in $^3P$\end{tabular} \\ \hline \hline
    C & $2p^2$  & 0.612081 ($\times 1$) & \underline{0.999995} &          & 0.000005 \\
      &         & 0.529403 ($\times 5$) &          & \underline{0.999992} & 0.000008 \\
      &         & 0.474392 ($\times 5$) &          & 0.000008 & \underline{0.999992} \\
      &         & 0.474175 ($\times 3$) &          &          & \underline{1.000000} \\
      &         & 0.474065 ($\times 1$) & 0.000005 &          & \underline{0.999995} \\ \hline
   Pb & $6p^2$  & 0.335963 ($\times 1$) & \underline{0.909208} &          & 0.090792 \\
      &         & 0.284981 ($\times 5$) &          & \underline{0.724683} & 0.275317 \\
      &         & 0.240833 ($\times 5$) &          & 0.275317 & \underline{0.724683} \\
      &         & 0.225100 ($\times 3$) &          &          & \underline{1.000000} \\
      &         & 0.198693 ($\times 1$) & 0.090792 &          & \underline{0.909208} \\
  \hline
\end{tabular}
\end{center}
\end{table}

If you watch carefully and compare Table~\ref{table:char}
with Fig.~\ref{fig:cvspb}, you will notice that the terms are
mixed if they have the same $J$. Maybe you also noticed the interesting ``1.000000''
which never mixes with the others. That is because the vectors are from the
term with a unique $J$.

This table of characters directly indicates how strongly
are the eigen-states mixed among different multiplet terms (different
angular momenta). It again evidenced that with spin-orbit effect,
the eigen-states in carbon (light atom) are slightly mixed with different multiplet terms,
but the eigen-states in lead (heavy atom) are strongly mixed with different multiplet terms.
I must point out that, the numerical diagonalization of $(H_U+H_\text{SO})$
can always give us a better estimation of the eigen-energies (because it uses
the complete basis in the open shell), which is especially important for heavy atoms.
Nevertheless, our construction of multiplet states and
the first order perturbation theory in spin-orbit coupling give us a very
important understanding of the problem and a deep insight into the physical system.


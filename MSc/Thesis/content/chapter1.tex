\chapter{Introduction}

\section{Many-body problem in quantum mechanics}
Imagine our solar system: it consists of a heavy sun with eight planets
orbiting around. In classical mechanics, the goal is to calculate the
position of each celestial object as a function of time. The motion of
the system is governed by Newton's second law: $\vec{F}=m\vec{a}$. While it is almost
impossible to solve the classical many-body problem analytically, the numerical
approach is rather straightforward. However, in quantum mechanics,
the many-body problem is a completely different story.

Scaling down to $10^{-10}$ meters, it is even philosophically profound that we
find similar structures in atomic systems as in our cosmic systems. An atom,
which consists of a heavy nucleus, with electric charge $Ze$, is surrounded by
$N$ electrons with charge $-e$. The system here is no longer governed by
Newton's second law, neither is our goal to compute the ``position'' of each
electron as a function of time. In quantum mechanics, the position of an
electron is no longer well defined. Instead, what we are looking for is the
electrons' wave function $\Psi(\vec{r}_1,\vec{r}_2,\ldots,\vec{r}_N)$ which
is the solution of the Schr\"{o}dinger equation\footnote{Strictly speaking,
Eqn.~(\ref{eq:Schr}) is called the time-independent Schr\"{o}dinger equation
and the solution $\Psi(\vec{r}_1,\vec{r}_2,\ldots,\vec{r}_N)$ is the
time-independent part of the wave function.}
\begin{equation} \label{eq:Schr}
H \Psi = E \Psi
\end{equation}
where $H$, the Hamiltonian of the system, is given by
\begin{equation} \label{eq:Hamit}
H = \sum_{i=1}^N \left[ -\frac{\hbar^2}{2m_e} \nabla_i^2 - \frac{1}{4\pi\epsilon_0} \frac{Ze^2}{r_i} \right] + \sum_{i<j}^N \frac{1}{4\pi\epsilon_0} \frac{e^2}{|\vec{r}_i - \vec{r}_j|}
\end{equation}
The term in the first sum represents the kinetic plus potential energy of the
$i$th electron, in the electric field of the nucleus. The last term, which
complicates the behavior of the system, describes the electron-electron
repulsions among all $N$ electrons (the constraint $i<j$ avoids double counting
over electron pairs).

In neither classical nor quantum mechanics, can one solve many-body problems
exactly. But in quantum mechanics, even a numerical solution is extremely
difficult to obtain. The difficulty comes from the fact that
a many-body wave function $\Psi(\vec{r}_1,\vec{r}_2,\ldots,\vec{r}_N)$ is
an object with dimension $3N$. One must generate a mesh grid with $3N$-dimension
to represent the wave function numerically and such a gigantic data is not
possible to be stored on an ordinary hard disk. Consequently, different methods
have been developed to simplify the problem and to treat the system approximately.

\section{Atomic units}
Since we are going to solve our problems numerically, it is
useful to pay attention to the choice of units. Certainly, we can use SI units,
but the scales would be inconvenient. For example, in SI units, the reduced
Planck constant reads $\hbar=1.054572\times10^{-34} \mathrm{J \cdot s}$, which
is a crazy number from a computational point of view. Hence, we employ atomic
units (a.u.), namely,
\begin{alignat*}{3}
& \text{Length:}\quad && 1\;\mathrm{a_0} && \approx 5.2918\times10^{-11}\;\mathrm{m}  \\
& \text{Mass:}\quad   && 1\;\mathrm{m_e} && \approx 9.1095\times10^{-31}\;\mathrm{kg} \\
& \text{Time:}\quad   && 1\;\mathrm{t_0} && \approx 2.4189\times10^{-17}\;\mathrm{s}  \\
& \text{Charge:}\quad && 1\;\mathrm{e}   && \approx 1.6022\times10^{-19}\;\mathrm{C}
\end{alignat*}
which are deliberately chosen such that
\begin{align}
\begin{split}
\hbar          & = 1\;\mathrm{a_0^2 m_e t_0^{-1}} \\
m_e            & = 1\;\mathrm{m_e} \\
e              & = 1\;\mathrm{e} \\
4\pi\epsilon_0 & = 1\;\mathrm{a_0^{-3} m_e t_0^2 e^2}
\end{split}
\end{align}
By adopting atomic units, the Hamiltonian in Eqn.~(\ref{eq:Hamit}) simplifies to
\begin{equation} \label{eq:HamitSimp}
H = \sum_{i=1}^N \left[ -\frac{1}{2} \nabla_i^2 - \frac{Z}{r_i} \right] + \sum_{i<j}^N \frac{1}{|\vec{r}_i - \vec{r}_j|}
\end{equation}
With the choice of atomic units, distances are given in units of the
Bohr radius ($\mathrm{a_0}$) and energies in Hartree ($\mathrm{a_0^2 m_e t_0^{-2}}$)
\begin{equation} \label{eq:EHartree}
  1\;\mathrm{Hartree}
= 2\;\mathrm{Rydberg}
\approx 27.2114\;\mathrm{eV}
\approx 4.3597\times10^{-18}\;\mathrm{J}
\end{equation}
From now on, we will keep using atomic units to simplify our equations and discussions.

\section{Convention of notations}
I will try to make the notations in my thesis as consistent as possible
to avoid confusions and to make the discussion clear.

For a many-electron wave function, we use capital psi
\begin{equation*}
\Psi(\vec{r}_1,\vec{r}_2,\ldots,\vec{r}_N)
\end{equation*}
For a one-electron wave function, we use lower case phi
\begin{equation*}
\varphi(\vec{r})
\end{equation*}
%
For a Slater determinant, we use upper case phi
\begin{equation*}
\Phi_{\alpha_1\alpha_2\cdots\alpha_N}(\vec{r}_1,\vec{r}_2,\ldots,\vec{r}_N)
\end{equation*}
%
Slater determinants are constructed from one-electron wave functions. They
are basis functions of anti-symmetric many-electron wave functions.
\begin{equation} \label{eq:introSD}
\Phi_{\alpha_1\cdots\alpha_N}(\vec{r}_1,\ldots,\vec{r}_N) = \frac{1}{\sqrt{N!}}
\begin{vmatrix}
\varphi_{\alpha_1}(\vec{r}_1) & \varphi_{\alpha_2}(\vec{r}_1) & \cdots & \varphi_{\alpha_N}(\vec{r}_1) \\
\varphi_{\alpha_1}(\vec{r}_2) & \varphi_{\alpha_2}(\vec{r}_2) & \cdots & \varphi_{\alpha_N}(\vec{r}_2) \\
\vdots & \vdots & \ddots & \vdots \\
\varphi_{\alpha_1}(\vec{r}_N) & \varphi_{\alpha_2}(\vec{r}_N) & \cdots & \varphi_{\alpha_N}(\vec{r}_N)
\end{vmatrix}
\end{equation}
%
Sometimes you hear people say, ``A many-electron wave function is a Slater determinant.''
That is wrong. A many-electron wave function is in general a linear combination of
Slater determinants, not just one (although it could be).
Nevertheless, this real-space representation of Slater determinant
in Eqn.~(\ref{eq:introSD}) will not appear in our discussion
(it only appears in Appendix~\ref{app:B}). Slater determinants will
be represented in the form of second quantization\footnote{Second quantization is
a very convenient ``algebra'' for handling many-body states. A brief discussion
of second quantization is given in Appendix~\ref{app:B}.} when later we
construct our multiplet states.




\chapter{Construction of multiplet states} \label{ch:5}

\section{Setting up the basis and Hamiltonian}
Oh, this is probably the most crucial step. All of our efforts spent so far on the
Slater-Condon parameters and the Gaunt coefficients are aimed at calculating
the matrix elements
\begin{equation} \label{eq:Uelem}
U_{\alpha\beta\gamma\delta} =
\delta_{\sigma_1\sigma_4} \delta_{\sigma_2\sigma_3}
\sum_{k(\text{sum rule})} \frac{4\pi}{2k+1} R^{(k)}(n_1l_1,n_2l_2,n_3l_3,n_4l_4)
A^{(k)}(l_1m_1,l_2m_2,l_3m_3,l_4m_4)
\end{equation}
so that we can build up the Coulomb repulsion Hamiltonian.

We called $U_{\alpha\beta\gamma\delta}$ the matrix element.
But as you might have noticed already, it has four indices
$\alpha$, $\beta$, $\gamma$ and $\delta$. It is not really a ``matrix'' in our
common sense. Indeed, $U_{\alpha\beta\gamma\delta}$ is not the matrix that we will
directly work with. The actual matrix we will be using is the 
matrix representation of the Hamiltonian $H_U$. Now, to obtain the matrix
representation, we must ask ourselves what basis do we work with.

Our first step is to set up a basis in the electron configuration space.
As a demonstration, we will do a case study for a $p^2$ orbital.
According to Pauli exclusion principle, one $p$ shell can at most contain
6 electrons with orbital and spin angular projection momenta $m=1,0,-1$ and
$\sigma=\uparrow,\downarrow$, respectively. One can think the problem as
6 boxes where we can put (up to) 6 electrons (see the diagram below).
\vspace{-0.5em}
\begin{equation*}
\begin{array}{c|c|c|c|}
\multicolumn{1}{c}{} & \multicolumn{1}{c}{\phantom{,}1\phantom{,}} & \multicolumn{1}{c}{\phantom{,}0\phantom{,}} & \multicolumn{1}{c}{-1} \\ \cline{2-4}
\uparrow &  &  &  \\ \cline{2-4}
\downarrow &  &  &  \\
\cline{2-4}
\end{array}
\end{equation*}
In our problem we have 2 electrons. How many possible ways are there
to put them into our boxes? This is simply a combinatorics problem.
\begin{center}
\vspace{-0.5em}
\begin{tabular}{|c|c|c|}
\hline
$\bullet$ & $\bullet$ & $\phantom{\bullet}$ \\ \hline
 &  &  \\
\hline
\end{tabular}
\begin{tabular}{|c|c|c|}
\hline
$\bullet$ & $\phantom{\bullet}$ & $\bullet$ \\ \hline
 &  &  \\
\hline
\end{tabular}
\begin{tabular}{|c|c|c|}
\hline
$\bullet$ & $\phantom{\bullet}$ & $\phantom{\bullet}$ \\ \hline
$\bullet$ &  &  \\
\hline
\end{tabular}
\begin{tabular}{|c|c|c|}
\hline
$\bullet$ & $\phantom{\bullet}$ & $\phantom{\bullet}$ \\ \hline
 & $\bullet$ &  \\
\hline
\end{tabular}
\begin{tabular}{|c|c|c|}
\hline
$\bullet$ & $\phantom{\bullet}$ & $\phantom{\bullet}$ \\ \hline
 &  & $\bullet$ \\
\hline
\end{tabular} \\
\vspace{0.5em}
\begin{tabular}{|c|c|c|}
\hline
$\phantom{\bullet}$ & $\bullet$ & $\bullet$ \\ \hline
 &  &  \\
\hline
\end{tabular}
\begin{tabular}{|c|c|c|}
\hline
$\phantom{\bullet}$ & $\bullet$ & $\phantom{\bullet}$ \\ \hline
$\bullet$ &  &  \\
\hline
\end{tabular}
\begin{tabular}{|c|c|c|}
\hline
$\phantom{\bullet}$ & $\bullet$ & $\phantom{\bullet}$ \\ \hline
 & $\bullet$ &  \\
\hline
\end{tabular}
\begin{tabular}{|c|c|c|}
\hline
$\phantom{\bullet}$ & $\bullet$ & $\phantom{\bullet}$ \\ \hline
 &  & $\bullet$ \\
\hline
\end{tabular}
\begin{tabular}{|c|c|c|}
\hline
$\phantom{\bullet}$ & $\phantom{\bullet}$ & $\bullet$ \\ \hline
$\bullet$ &  &  \\
\hline
\end{tabular} \\
\vspace{0.5em}
\begin{tabular}{|c|c|c|}
\hline
$\phantom{\bullet}$ & $\phantom{\bullet}$ & $\bullet$ \\ \hline
 & $\bullet$ &  \\
\hline
\end{tabular}
\begin{tabular}{|c|c|c|}
\hline
$\phantom{\bullet}$ & $\phantom{\bullet}$ & $\bullet$ \\ \hline
 &  & $\bullet$ \\
\hline
\end{tabular}
\begin{tabular}{|c|c|c|}
\hline
$\phantom{\bullet}$ & $\phantom{\bullet}$ & $\phantom{\bullet}$ \\ \hline
$\bullet$ & $\bullet$ &  \\
\hline
\end{tabular}
\begin{tabular}{|c|c|c|}
\hline
$\phantom{\bullet}$ & $\phantom{\bullet}$ & $\phantom{\bullet}$ \\ \hline
$\bullet$ &  & $\bullet$ \\
\hline
\end{tabular}
\begin{tabular}{|c|c|c|}
\hline
$\phantom{\bullet}$ & $\phantom{\bullet}$ & $\phantom{\bullet}$ \\ \hline
 & $\bullet$ & $\bullet$ \\
\hline
\end{tabular}
\end{center}
6 boxes put 2 electrons, that is ``6 choose 2''
\begin{equation*}
\binom{6}{2} = \frac{6\times5}{2!} = 15
\end{equation*}
possible ways. They are our 15 basis states (or basis vectors\footnote{By calling ``states''
or ``vectors'', we really mean the same. When we say a ``state'', it is more from a
quantum mechanics perspective that ``a state of electrons''. When we use ``vector'',
we more emphasize from the linear algebra point of view.}) of electrons in the configuration
space. Everyone says that, but what does it mean? Imagine you put two electrons
into the shell: Of course we can arrange them in the ways as in the diagram, but
electrons can do things peculiar, they can form linear combinations of those basis
states.\footnote{Linear combinations of states are also valid solutions of the Schr\"{o}dinger equation.}
Those 15 basis states (15 vectors) span a space that the electrons can live in. Recall our Cartesian
coordinate, we had 3 basis vectors called $\hat{\vec{x}}$, $\hat{\vec{y}}$ and $\hat{\vec{z}}$.
Similarly, we can also assign names to those 15 basis vectors, for example, from $\hat{\vec{a}}$
to $\hat{\vec{o}}$ (if I counted correctly). But a more convenient and sophisticated
naming method is the bit representation.
\begin{equation*}
\Ket{-----\,-}
\end{equation*}
Here we have 6 sites, if a site is occupied by an electron, we put a ``1'',
otherwise, we put a ``0'' (they are called occupation numbers). For example,
\begin{equation}
\begin{array}{|c|c|c|}
\hline
\bullet & \phantom{\bullet} & \phantom{\bullet} \\ \hline
 & \bullet &  \\
\hline
\end{array}
\equiv \Ket{100010}
\end{equation}
%
Therefore, our 15 basis states are:
\begin{equation*}
\begin{array}{c c c c c}
\Ket{110000} & \Ket{101000} & \Ket{100100} & \Ket{100010} & \Ket{100001} \\
\Ket{011000} & \Ket{010100} & \Ket{010010} & \Ket{010001} & \Ket{001100} \\
\Ket{001010} & \Ket{001001} & \Ket{000110} & \Ket{000101} & \Ket{000011}
\end{array}
\end{equation*}
%
The bit representation is convenient in the sense that they can be easily
stored in the computer. For instance, a binary number 100010 is nothing but a 34 in decimal.
To store $\Ket{100010}$ we simply store an integer 34 in the computer.
Yet, the pictorial meaning can be easily seen from its binary format.

A simple algorithm of the basis setup for a given shell \texttt{l} and a
given number of electrons \texttt{Ne} is shown Algorithm~\ref{alg:basis}.
Notice that the function \texttt{countBit()} counts for the number
of 1's in a given binary number.

After setting up the basis, we are now able to build the matrix representation of the Hamiltonian.
To obtain one entry at row $i$ and column $j$, it means to evaluate the matrix
element between two basis state $\Ket{i}$ and $\Ket{j}$.
\begin{equation} \label{eq:Huij}
\Bra{i} H_U \Ket{j} = 
\Bra{i} \frac{1}{2} \sum_{\alpha,\beta,\gamma,\delta} U_{\alpha\beta\gamma\delta} c_\alpha^\dag c_\beta^\dag c_\gamma c_\delta \Ket{j}
\end{equation}
%
It is not trivial at all to evaluate Eqn.~(\ref{eq:Huij}). One needs to implement a program
that loops over all six indices ($i,j,\alpha,\beta,\gamma,\delta$). For each pair of ($i,j$),
we sum up all the corresponding $U_{\alpha\beta\gamma\delta}$ to obtain the entry $\Bra{i}H_U\Ket{j}$.
Notice that the factor $1/2$ in front is used to compensate the double counting over
electron pairs. A simplified algorithm is shown in Algorithm~\ref{alg:Huij}.

\begin{algorithm}[h!]
\caption{Set up basis}\label{alg:basis}
\begin{algorithmic}[1]
\Function{Basis}{$l$, $N_e$}
\State $N_{\text{site}} \gets 4*l+2$
\State $dim \gets$ \Call{Binomial}{$N_{\text{site}}$, $N_e$}
\State $index_i \gets 0$
\For {$conf_i \gets 0$ to $2^{N_{\text{site}}}$}
\If {\Call{countBit}{$conf_i$} $=N_e$}
\State $index[conf_i] \gets index_i$
\State $conf[index_i] \gets conf_i$
\State $index_i \gets index_i + 1$
\EndIf
\EndFor
\State $basis.dim \gets dim$
\State $basis.index \gets index$
\State $basis.conf \gets conf$
\State \Return $basis$
\EndFunction
\end{algorithmic}
\end{algorithm}

\begin{algorithm}[h!]
\caption{Set up Hamiltonian}\label{alg:Huij}
\begin{algorithmic}[1]
\Function{Hamiltonian}{$basis$}
\State $dim \gets basis.dim$
\For {$i \gets 0$ to $dim$}
\State $conf_i \gets basis.conf[i]$
\ForAll {$\alpha$}
\If {\Call{isBit}{$\alpha$, $conf_i$}}
\State $conf_\alpha \gets$ \Call{clearBit}{$\alpha$, $conf_i$}
\ForAll {$\beta$}
\If {\Call{isBit}{$\beta$, $conf_\alpha$}}
\State $conf_\beta \gets$ \Call{clearBit}{$\beta$, $conf_\alpha$}
\ForAll {$\gamma$}
\If {\Call{!isBit}{$\gamma$, $conf_\beta$}}
\State $conf_\gamma \gets$ \Call{setBit}{$\gamma$, $conf_\beta$}
\ForAll {$\delta$}
\If {\Call{!isBit}{$\delta$, $conf_\gamma$}}
\State $conf_\delta \gets$ \Call{setBit}{$\delta$, $conf_\gamma$}
\State $conf_j \gets conf_\delta$
\State $j \gets basis.index[conf_j]$
\State $H_U[i,j] \gets H_U[i,j] + fsign*0.5*U_{\alpha\beta\gamma\delta}$
\EndIf
\EndFor
\EndIf
\EndFor
\EndIf
\EndFor
\EndIf
\EndFor
\EndFor
\State \Return $H_U$
\EndFunction
\end{algorithmic}
\end{algorithm}

The function \texttt{clearBit()} clears a bit in a given configuration. For example,
\texttt{clearBit(2,111111)} returns \texttt{110111} (in binary). Similarly,
\texttt{setBit()} sets a bit in a given configuration. The function \texttt{isBit()}
tests whether a specific site is occupied by an electron or not, from which we
know if we can clear or set a bit on that site. It is (always) tricky that
we have to pay extra attention to the fermi-sign \texttt{fsign}. Every time
we call a function \texttt{clearBit()} or \texttt{setBit()}, it means that we applied
an electron annihilation or creation operator. But when we apply an annihilation or
creation operator, a fermi-sign ($\pm1$) is involved \cite{GBK}.

\section{Construction of eigen-states}
Our problem is almost solved.
What we want to know ultimately are the eigen-vectors and eigen-energies
of the Hamiltonian. But how big is this business? Since we have the matrix,
to get the eigen-vectors and eigen-energies, we can simply throw it into a matrix
diagonalization solver (e.g.\ \texttt{Lapack}), then we are done.
But, is this the end of our story?

We continue our case study for the $p^2$ orbital. In our previous section, we set
the basis up by 15 basis vectors (let's first sort the basis vectors
in an ascending order of binary numbers). Under this basis, we built up a 15-by-15
Hamiltonian. It has the following pattern:
\begin{equation*}
H_U =
\begin{blockarray}{cccccccccccccccc}
\begin{block}{[ccccccccccccccc]c}
\otimes &  &  &  &  &  &  &  &  &  &  &  &  &  &  & \Ket{000011} \\
 & \otimes &  &  &  &  &  &  &  &  &  &  &  &  &  & \Ket{000101} \\
 &  & \otimes &  &  &  &  &  &  &  &  &  &  &  &  & \Ket{000110} \\
 &  &  & \otimes &  &  &  &  &  &  &  &  &  &  &  & \Ket{001001} \\
 &  &  &  & \otimes &  & \otimes &  &  &  &  &  &  &  &  & \Ket{001010} \\
 &  &  &  &  & \otimes &  & \otimes &  &  & \otimes &  &  &  &  & \Ket{001100} \\
 &  &  &  & \otimes &  & \otimes &  &  &  &  &  &  &  &  & \Ket{010001} \\
 &  &  &  &  & \otimes &  & \otimes &  &  & \otimes &  &  &  &  & \Ket{010010} \\
 &  &  &  &  &  &  &  & \otimes &  &  & \otimes &  &  &  & \Ket{010100} \\
 &  &  &  &  &  &  &  &  & \otimes &  &  &  &  &  & \Ket{011000} \\
 &  &  &  &  & \otimes &  & \otimes &  &  & \otimes &  &  &  &  & \Ket{100001} \\
 &  &  &  &  &  &  &  & \otimes &  &  & \otimes &  &  &  & \Ket{100010} \\
 &  &  &  &  &  &  &  &  &  &  &  & \otimes &  &  & \Ket{100100} \\
 &  &  &  &  &  &  &  &  &  &  &  &  & \otimes &  & \Ket{101000} \\
 &  &  &  &  &  &  &  &  &  &  &  &  &  & \otimes & \Ket{110000} \\
\end{block}
\end{blockarray}
\end{equation*}
The empty elements are killed by the $\delta_{\sigma_1\sigma_4}\delta_{\sigma_2\sigma_3}$
in Eqn.~(\ref{eq:Uelem}) (conservation of spin angular momentum)
and the vanishing Gaunt coefficients. Recall Eqn.~(\ref{eq:AkBKone}),
the sum rule requires $m_1+m_2 = m_3+m_4$ (conservation of orbital angular momentum).
At this stage, we can throw $H_U$ into a matrix diagonalization solver,
then we will obtain 15 eigen-energies plus 15 corresponding eigen-vectors.
For people who are only interested in the eigen-energies (spectral lines)
of the system, calling a matrix diagonalization solver will be enough
to solve the problem. But if one wants to understand more physics about the system,
asking, ``What exactly are the quantum numbers of those eigen-states?'', our solutions
from the ``brute-force'' diagonalization cannot explain. As we shall find out later,
it turns out that those 15 eigen-energies are highly
degenerate. Among those 15 energies, we will obtain something like:
$9\times E_1$, $5\times E_2$, $1\times E_3$.
Now the problem is, if the eigen-energies are degenerate, the corresponding
eigen-vectors are not uniquely defined. They live in a space where any
linear combinations can be eigen-vectors. Thus it is difficult to address
each eigen-vector by a specific set of quantum numbers.
This high degeneracy that we claimed seems to be a mystery.
In fact, the reason behind is that there is a special symmetry in the Hamiltonian $H_U$.
Based on this symmetry, we can use a special technique to diagonalize this matrix
without calling a matrix diagonalization routine like \texttt{Lapack}.
More importantly, with this technique, eigen-vectors can be identified uniquely with beautiful quantum numbers.
The idea is to use the angular momentum ladder operators. First, we need to prove that
the Hamiltonian $H_U$ commutes with the total angular momenta $\vec{L}$ and $\vec{S}$.

\paragraph{Proof:} By definition, the total angular momenta are
$\vec{L}=\sum_{i=1}^N\boldsymbol{\ell}_i$ and $\vec{S}=\sum_{i=1}^N\vec{s}_i$.

$[H_U,\,\vec{S}]=0$ is trivial, since
\begin{equation}
H_U = \sum_{i<j}^N \frac{1}{|\vec{r}_i - \vec{r}_j|}
\end{equation}
has no dependence on spin.

$[H_U,\,\vec{L}]=0$ is a bit more difficult. We need
\begin{equation}
\boldsymbol{\ell}_i = -i\vec{r}_i \times \boldsymbol{\nabla}_i
\end{equation}
Now,
\begin{align}
\left[\frac{1}{|\vec{r}_i-\vec{r}_j|},\,\boldsymbol{\ell}_i \right] & = \frac{1}{|\vec{r}_i-\vec{r}_j|}(-i\vec{r}_i \times \boldsymbol{\nabla}_i) - (-i\vec{r}_i \times \boldsymbol{\nabla}_i)\frac{1}{|\vec{r}_i-\vec{r}_j|} \nonumber \\
& = i\vec{r}_i \times \left( \boldsymbol{\nabla}_i \frac{1}{|\vec{r}_i-\vec{r}_j|} \right) \nonumber \\
& = i\vec{r}_i \times \left( -\frac{\vec{r}_i-\vec{r}_j}{|\vec{r}_i-\vec{r}_j|^3} \right) \nonumber \\
& = i \frac{\vec{r}_i \times \vec{r}_j}{|\vec{r}_i-\vec{r}_j|^3} \nonumber \\
& = i\vec{r}_j \times \frac{\vec{r}_j-\vec{r}_i}{|\vec{r}_j-\vec{r}_i|^3} \nonumber \\
& = -i\vec{r}_j \times \left( \boldsymbol{\nabla}_j \frac{1}{|\vec{r}_j-\vec{r}_i|} \right)
= -\left[\frac{1}{|\vec{r}_i-\vec{r}_j|},\,\boldsymbol{\ell}_j \right]
\end{align}
Therefore,
\begin{align}
\left[\frac{1}{|\vec{r}_i-\vec{r}_j|},\,\boldsymbol{\ell}_i+\boldsymbol{\ell}_j \right] & = 0 \\
\left[\sum_{i<j}^N \frac{1}{|\vec{r}_i - \vec{r}_j|},\,\sum_{i=0}^N\boldsymbol{\ell}_i \right] & = 0 \qquad \text{Q.E.D.}
\end{align}

Given that $[H_U,\,\vec{L}]=0$ and $[H_U,\,\vec{S}]=0$,
we have the following relations:
\begin{equation*}
[H_U,\,L^2]=0 \qquad [H_U,\,L_z]=0 \qquad
[H_U,\,S^2]=0 \qquad [H_U,\,S_z]=0
\end{equation*}
%
As a result, an eigen-vector of $H_U$ is simultaneously
eigen-vectors of $L^2$, $L_z$, $S^2$, $S_z$. Thus, we can denote
an eigen-vector as $\Ket{L, M_L, S, M_S}$.\footnote{At this stage, we
denote an eigen-vector with the four quantum numbers $\Ket{L, M_L, S, M_S}$,
which is normally sufficient.
However, in the next section, we will find the situation in which $\Ket{L, M_L, S, M_S}$
is not able to address an eigen-vector uniquely. Then we would require
an additional quantum number (called seniority number) to address the
eigen-vector.}
The commutation relations also lead to $[H_U,\,L_\pm]=0$ and $[H_U,\,S_\pm]=0$.
Therefore, for a given eigen-vector $\Ket{L, M_L, S, M_S}$ of $H_U$,
we can find new eigen-vectors using the ladder operators, for example,
\begin{align} \label{eq:LmOp}
\begin{split}
L_- H_U \Ket{L, M_L, S, M_S} & = L_- E \Ket{L, M_L, S, M_S} \\
H_U L_- \Ket{L, M_L, S, M_S} & = E L_- \Ket{L, M_L, S, M_S} \\
H_U \Ket{L, M_L-1, S, M_S} & = E \Ket{L, M_L-1, S, M_S}
\end{split}
\end{align}
Similarly,
\begin{align} \label{eq:SmOp}
\begin{split}
S_- H_U \Ket{L, M_L, S, M_S} & = S_- E \Ket{L, M_L, S, M_S} \\
H_U S_- \Ket{L, M_L, S, M_S} & = E S_- \Ket{L, M_L, S, M_S} \\
H_U \Ket{L, M_L, S, M_S-1} & = E \Ket{L, M_L, S, M_S-1}
\end{split}
\end{align}
%
The idea is,
\begin{equation}
\boxed{
\begin{aligned}
& \textbf{Starting from a leading vector, we can construct subsequent} \\
& \textbf{vectors by applying ladder operators.}
\end{aligned}
}
\end{equation}
This is the technique we will use to diagonalize our Hamiltonian.

In Cartesian coordinates, we are used to arrange the basis vectors as
$[\hat{\vec{x}},\,\hat{\vec{y}},\,\hat{\vec{z}}]$. But it really doesn't hurt if we
do $[\hat{\vec{y}},\,\hat{\vec{x}},\,\hat{\vec{z}}]$. Previously we arranged the basis
vectors in an ascending order of binary numbers. But we can also arrange
the order such that we see clearly the pattern of the Hamiltonian.
\begin{equation*}
H_U =
\begin{blockarray}{ccccccccccccccccc}
 &  &  &  &  &  &  &  &  &  &  &  &  &  &  &  & \Ket{M_{L}\,M_{S}} \\
\begin{block}{[ccccccccccccccc]cc}
\otimes &  &  &  &  &  &  &  &  &  &  &  &  &  &  & \Ket{100100} & \Ket{{\phantom{-}2}\ {\phantom{-}0}} \\
 & \otimes &  &  &  &  &  &  &  &  &  &  &  &  &  & \Ket{110000} & \Ket{{\phantom{-}1}\ {\phantom{-}1}} \\
 &  & \otimes & \otimes &  &  &  &  &  &  &  &  &  &  &  & \Ket{010100} & \Ket{{\phantom{-}1}\ {\phantom{-}0}} \\
 &  & \otimes & \otimes &  &  &  &  &  &  &  &  &  &  &  & \Ket{100010} & \Ket{{\phantom{-}1}\ {\phantom{-}0}} \\
 &  &  &  & \otimes &  &  &  &  &  &  &  &  &  &  & \Ket{000110} & \Ket{{\phantom{-}1}\ {-1}} \\
 &  &  &  &  & \otimes &  &  &  &  &  &  &  &  &  & \Ket{101000} & \Ket{{\phantom{-}0}\ {\phantom{-}1}} \\
 &  &  &  &  &  & \otimes & \otimes & \otimes &  &  &  &  &  &  & \Ket{001100} & \Ket{{\phantom{-}0}\ {\phantom{-}0}} \\
 &  &  &  &  &  & \otimes & \otimes & \otimes &  &  &  &  &  &  & \Ket{010010} & \Ket{{\phantom{-}0}\ {\phantom{-}0}} \\
 &  &  &  &  &  & \otimes & \otimes & \otimes &  &  &  &  &  &  & \Ket{100001} & \Ket{{\phantom{-}0}\ {\phantom{-}0}} \\
 &  &  &  &  &  &  &  &  & \otimes &  &  &  &  &  & \Ket{000101} & \Ket{{\phantom{-}0}\ {-1}} \\
 &  &  &  &  &  &  &  &  &  & \otimes &  &  &  &  & \Ket{011000} & \Ket{{-1}\ {\phantom{-}1}} \\
 &  &  &  &  &  &  &  &  &  &  & \otimes & \otimes &  &  & \Ket{001010} & \Ket{{-1}\ {\phantom{-}0}} \\
 &  &  &  &  &  &  &  &  &  &  & \otimes & \otimes &  &  & \Ket{010001} & \Ket{{-1}\ {\phantom{-}0}} \\
 &  &  &  &  &  &  &  &  &  &  &  &  & \otimes &  & \Ket{000011} & \Ket{{-1}\ {-1}} \\
  &  &  &  &  &  &  &  &  &  &  &  &  &  & \otimes & \Ket{001001} & \Ket{{-2}\ {\phantom{-}0}} \\
\end{block}
\end{blockarray}
\end{equation*}
%
Why this pattern? Consider the configurations $\Ket{001100}$, $\Ket{010010}$ and $\Ket{100001}$:
\begin{center}
\begin{tabular}{|c|c|c|}
\hline
$\phantom{\bullet}$ & $\phantom{\bullet}$ & $\bullet$ \\ \hline
$\bullet$ &  &  \\
\hline
\end{tabular}
\begin{tabular}{|c|c|c|}
\hline
$\phantom{\bullet}$ & $\bullet$ & $\phantom{\bullet}$ \\ \hline
 & $\bullet$ &  \\
\hline
\end{tabular}
\begin{tabular}{|c|c|c|}
\hline
$\bullet$ & $\phantom{\bullet}$ & $\phantom{\bullet}$ \\ \hline
 &  & $\bullet$ \\
\hline
\end{tabular}
\end{center}
Those configurations have the same total $M_L = 0$ and total $M_S = 0$.
Because of conservation of angular momentum, the electrons can move freely
among those three configurations. This gives us the 3-by-3 block in the middle
of the Hamiltonian. But a configuration like $\Ket{100100}$
\begin{center}
\begin{tabular}{|c|c|c|}
\hline
$\bullet$ & $\phantom{\bullet}$ & $\phantom{\bullet}$ \\ \hline
$\bullet$ &  &  \\
\hline
\end{tabular}
\end{center}
has no where to go. It has the maximum $M_L = 1+1 = 2$.
Any other configuration will give a lower $M_L$. Consequently, there is only
one element in the corresponding row and column in the matrix.
Hence, the state $\Ket{100100}$ is an eigen-state of the Hamiltonian.
But, as you might have noticed already, $\Ket{100100}$ is not the only basis
state with the unique $\Ket{M_{L}\,M_{S}}$. It is extremely useful to make
a table to count how many basis states are there for a given
$\Ket{M_{L}\,M_{S}}$.
\begin{center}
\begin{tabular}{c c|r c r}
      &  &  & $M_S$ & \\
      &  & $-1$ & $0$ & $1$ \\ \hline
      & $\phantom{-}2$ & 0 & 1 & 0 \\
      & $\phantom{-}1$ & 1 & 2 & 1 \\
$M_L$ & $\phantom{-}0$ & 1 & 3 & 1 \\
      & $-1$ & 1 & 2 & 1 \\
      & $-2$ & 0 & 1 & 0 \\
\end{tabular}
\end{center}

We labeled our basis states using two quantum numbers $M_L$ and $M_S$ and we found out
that there could be more than one basis state sharing the same $M_L$ and $M_S$.
An eigen-vector of the Hamiltonian is, in general, a linear combination of the basis states
with the same $M_L$ and $M_S$. If we are lucky, we find some basis states with unique
quantum numbers. Then those states are eigen-vectors by themselves.

Here comes the technique for constructing the eigen-vectors.
We start from the basis state with the largest $M_S$ and the corresponding largest
$M_L$ (which is unique):
\begin{center}
\begin{tabular}{c c|r c c}
      &  &  & $M_S$ & \\
      &  & $-1$ & $0$ & $1$ \\ \hline
      & $\phantom{-}2$ & 0 & 1 & 0 \\
      & $\phantom{-}1$ & 1 & 2 & $\boxed{1}$ \\
$M_L$ & $\phantom{-}0$ & 1 & 3 & 1 \\
      & $-1$ & 1 & 2 & 1 \\
      & $-2$ & 0 & 1 & 0 \\
\end{tabular}
\end{center}

This gives us an eigen-vector $\Ket{1,1,1,1}$. By applying $L_-$ and $S_-$ operators,
we obtain 
\begin{center}
\begin{tabular}{c c|r c c r r r}
      &  &  & $M_S$ &  &  &  \\
      &  & $-1$ & $0$ & $1$ &  &  \\ \cline{1-5}
      & $\phantom{-}2$ & 0 & 1 & 0 &  &  &  \\
      & $\phantom{-}1$ & $\boxed{1}$ & $\boxed{2}$ & $\boxed{1}$ & $\quad\Ket{1,\phantom{-}1,1,-1}$ & $\Ket{1,\phantom{-}1,1,\phantom{-}0}$ & $\Ket{1,\phantom{-}1,1,\phantom{-}1}$ \\
$M_L$ & $\phantom{-}0$ & $\boxed{1}$ & $\boxed{3}$ & $\boxed{1}$ & $\quad\Ket{1,\phantom{-}0,1,-1}$ & $\Ket{1,\phantom{-}0,1,\phantom{-}0}$ & $\Ket{1,\phantom{-}0,1,\phantom{-}1}$ \\
      & $-1$ & $\boxed{1}$ & $\boxed{2}$ & $\boxed{1}$ & $\quad\Ket{1,-1,1,-1}$ & $\Ket{1,-1,1,\phantom{-}0}$ & $\Ket{1,-1,1,\phantom{-}1}$ \\
      & $-2$ & 0 & 1 & 0 &  &  &  \\
\end{tabular}
\end{center}

Since we have extracted one vector out of each entry, we decrement each entry by 1.
\begin{center}
\begin{tabular}{c c|r c c}
      &  &  & $M_S$ & \\
      &  & $-1$ & $0$ & $1$ \\ \hline
      & $\phantom{-}2$ & 0 & 1 & 0 \\
      & $\phantom{-}1$ & 0 & 1 & 0 \\
$M_L$ & $\phantom{-}0$ & 0 & 2 & 0 \\
      & $-1$ & 0 & 1 & 0 \\
      & $-2$ & 0 & 1 & 0 \\
\end{tabular}
\end{center}

Again, we start from the basis state with the largest $M_S$ and the corresponding
largest $M_L$:
\begin{center}
\begin{tabular}{c c|r c c}
      &  &  & $M_S$ & \\
      &  & $-1$ & $0$ & $1$ \\ \hline
      & $\phantom{-}2$ & 0 & $\boxed{1}$ & 0 \\
      & $\phantom{-}1$ & 0 & 1 & 0 \\
$M_L$ & $\phantom{-}0$ & 0 & 2 & 0 \\
      & $-1$ & 0 & 1 & 0 \\
      & $-2$ & 0 & 1 & 0 \\
\end{tabular}
\end{center}

This gives us an eigen-vector $\Ket{2,2,0,0}$. Now apply both $L_-$ and $S_-$
operators (in this case only $L_-$), we obtain
\begin{center}
\begin{tabular}{c c|r c c r}
      &  &  & $M_S$ &  & \\
      &  & $-1$ & $0$ & $1$ & \\ \cline{1-5}
      & $\phantom{-}2$ & 0 & $\boxed{1}$ & 0 & $\quad\Ket{2,\phantom{-}2,0,\phantom{-}0}$ \\
      & $\phantom{-}1$ & 0 & $\boxed{1}$ & 0 & $\quad\Ket{2,\phantom{-}1,0,\phantom{-}0}$ \\
$M_L$ & $\phantom{-}0$ & 0 & $\boxed{2}$ & 0 & $\quad\Ket{2,\phantom{-}0,0,\phantom{-}0}$ \\
      & $-1$ & 0 & $\boxed{1}$ & 0 & $\quad\Ket{2,-1,0,\phantom{-}0}$ \\
      & $-2$ & 0 & $\boxed{1}$ & 0 & $\quad\Ket{2,-2,0,\phantom{-}0}$ \\
\end{tabular}
\end{center}

Now we decrement each entry by 1.
\begin{center}
\begin{tabular}{c c|r c c}
      &  &  & $M_S$ & \\
      &  & $-1$ & $0$ & $1$ \\ \hline
      & $\phantom{-}2$ & 0 & 0 & 0 \\
      & $\phantom{-}1$ & 0 & 0 & 0 \\
$M_L$ & $\phantom{-}0$ & 0 & \boxed{1} & 0 \\
      & $-1$ & 0 & 0 & 0 \\
      & $-2$ & 0 & 0 & 0 \\
\end{tabular}
\end{center}

$\Ket{0,0,0,0}$ is the last eigen-vector in our example. It cannot be read off directly, since there are 3 basis
vectors corresponding to this $M_L$ and $M_S$. But two eigen-vectors have been constructed already, namely,
$\Ket{1,0,1,0}$ and $\Ket{2,0,0,0}$. The last one, $\Ket{0,0,0,0}$, should be obtained from the
orthogonality relation from the previous two eigen-vectors.

Once we applied a ladder operator $L_-$ or $S_-$, we changed only the ``projection'' quantum number
$M_L$ or $M_S$. The amplitude of angular momenta $L$ and $S$ are never changed by ladder operators.
From Eqns.~(\ref{eq:LmOp}) and (\ref{eq:SmOp}) we can conclude
that eigen-vectors with the same $L$ and $S$ have the same eigen-energy.
In other words, the eigen-vectors obtained from the ladder operators have degenerate
eigen-energies. If we review the previous example, we can summarize that there are three
groups of eigen-vectors:
\begin{center}
\begin{tabular}{|c c c|c|c|}
\hline
 & $^3P$ &  & $^1D$ & $^1S$  \\ \hline
 &  &  & $\Ket{2,\phantom{-}2,0,\phantom{-}0}$ & \\
 $\Ket{1,\phantom{-}1,1,-1}$ & $\Ket{1,\phantom{-}1,1,\phantom{-}0}$ & $\Ket{1,\phantom{-}1,1,\phantom{-}1}$ & $\Ket{2,\phantom{-}1,0,\phantom{-}0}$ & \\
 $\Ket{1,\phantom{-}0,1,-1}$ & $\Ket{1,\phantom{-}0,1,\phantom{-}0}$ & $\Ket{1,\phantom{-}0,1,\phantom{-}1}$ & $\Ket{2,\phantom{-}0,0,\phantom{-}0}$ & $\Ket{0,\phantom{-}0,0,\phantom{-}0}$ \\
 $\Ket{1,-1,1,-1}$ & $\Ket{1,-1,1,\phantom{-}0}$ & $\Ket{1,-1,1,\phantom{-}1}$ & $\Ket{2,-1,0,\phantom{-}0}$ & \\
 &  &  & $\Ket{2,-2,0,\phantom{-}0}$ & \\
\hline
\end{tabular}
\end{center}

Each group is called a multiplet. All vectors from the same group have the same eigen-energy
(this confirmed the ``$9\times E_1$, $5\times E_2$, $1\times E_3$'' mystery we claimed in
the beginning of this section). Eigen-energies are highly degenerate,
that is why ``multi''-plet. The convention of naming is the term symbol:
\begin{equation} \label{eq:mtp}
^{2S+1}L
\end{equation}
with
\begin{equation} \label{eq:mtpDegen}
\mathrm{degeneracy} = (2S+1)(2L+1)
\end{equation}
%
There is also a convention of pronouncing the multiplets. For example,
$^3P$, $^1D$ and $^1S$ are pronounced ``triplet pee'', ``singlet dee'' and
``singlet ess''. A table for mapping the superscript $2S+1$ (called multiplicity)
and their names:
\begin{center}
\begin{tabular}{|c|l|}
\hline
 $2S+1$ & name \\ \hline
 1 & singlet \\
 2 & doublet \\
 3 & triplet \\
 4 & quartet \\
 5 & quintet \\
 6 & sextet \\
 7 & septet \\
 8 & octet \\
 9 & nonet \\
\hline
\end{tabular}
\end{center}

The first few symbols of $L$ are: (oh, don't ask me why there is no ``$J$'')
\begin{center}
\begin{tabular}{|c|c c c c c c c c c c c c c c c c c|}
\hline
 $L$ & 0 & 1 & 2 & 3 & 4 & 5 & 6 & 7 & 8 & 9 & 10 & 11 & 12 & 13 & 14 & 15 & 16 \\ \hline
     & $S$ & $P$ & $D$ & $F$ & $G$ & $H$ & $I$ & $K$ & $L$ & $M$ & $N$ & $O$ & $Q$ & $R$ & $T$ & $U$ & $V$ \\
\hline
\end{tabular}
\end{center}

Of course, all eigen-vectors should be expressed in terms of our 15 basis states. It won't
tell us anything if we just have a name like $\Ket{2,1,0,0}$. Those expression are
obtained when we apply the ladder operators. For example, (vectors should be normalized)
\begin{align} \label{eq:LmBas}
L_-
\begin{array}{|c|c|c|}
\hline
\bullet & \phantom{\bullet} & \phantom{\bullet} \\ \hline
\bullet &  &  \\
\hline
\end{array} & =
\sqrt{(1+1)(1-1+1)}
\begin{array}{|c|c|c|}
\hline
\phantom{\bullet} & \bullet & \phantom{\bullet} \\ \hline
\bullet &  &  \\
\hline
\end{array} +
\sqrt{(1+1)(1-1+1)}
\begin{array}{|c|c|c|}
\hline
\bullet & \phantom{\bullet} & \phantom{\bullet} \\ \hline
 & \bullet &  \\
\hline
\end{array} \nonumber \\
L_-\Ket{2,2,0,0} & = \qquad \qquad \qquad \quad \,
\sqrt{2}\:
\begin{array}{|c|c|c|}
\hline
\phantom{\bullet} & \bullet & \phantom{\bullet} \\ \hline
\bullet &  &  \\
\hline
\end{array} + \qquad \qquad \qquad \quad \,
\sqrt{2}\:
\begin{array}{|c|c|c|}
\hline
\bullet & \phantom{\bullet} & \phantom{\bullet} \\ \hline
 & \bullet &  \\
\hline
\end{array} \nonumber \\
\Ket{2,1,0,0} & = \qquad \qquad \qquad \quad \,
\frac{1}{\sqrt{2}}
\begin{array}{|c|c|c|}
\hline
\phantom{\bullet} & \bullet & \phantom{\bullet} \\ \hline
\bullet &  &  \\
\hline
\end{array} + \qquad \qquad \qquad \quad \,
\frac{1}{\sqrt{2}}
\begin{array}{|c|c|c|}
\hline
\bullet & \phantom{\bullet} & \phantom{\bullet} \\ \hline
 & \bullet &  \\
\hline
\end{array}
\end{align}
%
In second quantization, Eqn~(\ref{eq:LmBas}) can be written as
\begin{equation}
\Ket{2,1,0,0} =
\frac{1}{\sqrt{2}} \left( c_{1\downarrow}^\dagger c_{0\uparrow}^\dagger
+ c_{0\downarrow}^\dagger c_{1\uparrow}^\dagger \right) \Ket{0}
\end{equation}
\vspace{1em}

We made the convention that the order of creation operators are arranged
according to the configuration:
\begin{equation}
\begin{array}{c|c|c|c|}
\multicolumn{1}{c}{} & \multicolumn{1}{c}{\phantom{,}1\phantom{,}} & \multicolumn{1}{c}{\phantom{,}0\phantom{,}} & \multicolumn{1}{c}{-1} \\ \cline{2-4}
\uparrow & 1 & 2 & 3 \\ \cline{2-4}
\downarrow & 4 & 5 & 6 \\
\cline{2-4}
\multicolumn{1}{c}{}
\end{array} =
c_6^\dagger c_5^\dagger c_4^\dagger
c_3^\dagger c_2^\dagger c_1^\dagger
\Ket{0}
\end{equation}
%
Hence,
\begin{equation} \label{eq:createorder}
\begin{array}{|c|c|c|}
\hline
\bullet & \bullet & \bullet \\ \hline
\bullet & \bullet & \bullet \\
\hline
\end{array} =
c_{-1\downarrow}^\dagger c_{0\downarrow}^\dagger c_{1\downarrow}^\dagger
c_{-1\uparrow}^\dagger c_{0\uparrow}^\dagger c_{1\uparrow}^\dagger
\Ket{0}
\end{equation}

In summary, our 15 eigen-vectors are listed below:
\begin{equation*}
\renewcommand\arraystretch{1.8}
\begin{array}{|>{\displaystyle}c|>{\displaystyle}c >{\displaystyle}c >{\displaystyle}l|}
\hline
& |1,\phantom{-}1,1,\phantom{-}1\rangle & = & c_{0\uparrow}^\dagger c_{1\uparrow}^\dagger |0\rangle \\ 
& |1,\phantom{-}1,1,\phantom{-}0\rangle & = & \frac{1}{\sqrt{2}} \left( -c_{1\downarrow}^\dagger c_{0\uparrow}^\dagger +c_{0\downarrow}^\dagger c_{1\uparrow}^\dagger \right)|0\rangle \\ 
& |1,\phantom{-}1,1,-1\rangle & = & c_{0\downarrow}^\dagger c_{1\downarrow}^\dagger |0\rangle \\ 
& |1,\phantom{-}0,1,\phantom{-}1\rangle & = & c_{-1\uparrow}^\dagger c_{1\uparrow}^\dagger |0\rangle \\ 
^{3}P & |1,\phantom{-}0,1,\phantom{-}0\rangle & = & \frac{1}{\sqrt{2}} \left( -c_{1\downarrow}^\dagger c_{-1\uparrow}^\dagger +c_{-1\downarrow}^\dagger c_{1\uparrow}^\dagger \right)|0\rangle \\ 
& |1,\phantom{-}0,1,-1\rangle & = & c_{-1\downarrow}^\dagger c_{1\downarrow}^\dagger |0\rangle \\ 
& |1,-1,1,\phantom{-}1\rangle & = & c_{-1\uparrow}^\dagger c_{0\uparrow}^\dagger |0\rangle \\ 
& |1,-1,1,\phantom{-}0\rangle & = & \frac{1}{\sqrt{2}} \left( -c_{0\downarrow}^\dagger c_{-1\uparrow}^\dagger +c_{-1\downarrow}^\dagger c_{0\uparrow}^\dagger \right)|0\rangle \\ 
& |1,-1,1,-1\rangle & = & c_{-1\downarrow}^\dagger c_{0\downarrow}^\dagger |0\rangle \\
\hline
& |2,\phantom{-}2,0,\phantom{-}0\rangle & = & c_{1\downarrow}^\dagger c_{1\uparrow}^\dagger |0\rangle \\ 
& |2,\phantom{-}1,0,\phantom{-}0\rangle & = & \frac{1}{\sqrt{2}} \left( c_{1\downarrow}^\dagger c_{0\uparrow}^\dagger +c_{0\downarrow}^\dagger c_{1\uparrow}^\dagger \right)|0\rangle \\ 
^{1}D & |2,\phantom{-}0,0,\phantom{-}0\rangle & = & \frac{1}{\sqrt{6}} \left( c_{1\downarrow}^\dagger c_{-1\uparrow}^\dagger +2c_{0\downarrow}^\dagger c_{0\uparrow}^\dagger +c_{-1\downarrow}^\dagger c_{1\uparrow}^\dagger \right)|0\rangle \\ 
& |2,-1,0,\phantom{-}0\rangle & = & \frac{1}{\sqrt{2}} \left( c_{0\downarrow}^\dagger c_{-1\uparrow}^\dagger +c_{-1\downarrow}^\dagger c_{0\uparrow}^\dagger \right)|0\rangle \\ 
& |2,-2,0,\phantom{-}0\rangle & = & c_{-1\downarrow}^\dagger c_{-1\uparrow}^\dagger |0\rangle \\ 
\hline 
^{1}S & |0,\phantom{-}0,0,\phantom{-}0\rangle & = & \frac{1}{\sqrt{3}} \left( c_{1\downarrow}^\dagger c_{-1\uparrow}^\dagger -c_{0\downarrow}^\dagger c_{0\uparrow}^\dagger +c_{-1\downarrow}^\dagger c_{1\uparrow}^\dagger \right)|0\rangle \\
\hline
\end{array}
\end{equation*}

As a remark, we started from the basis state with the largest $M_S$ and the corresponding largest
$M_L$. Equivalently, we could also start from the largest $M_L$ and the corresponding largest
$M_S$, since they are both eigen-states. Actually, it is just a matter of taste that
we start from either
\begin{equation*}
\begin{array}{|c|c|c|}
\hline
\bullet & \bullet & \phantom{\bullet} \\ \hline
\phantom{\bullet} & \phantom{\bullet} & \phantom{\bullet} \\
\hline
\multicolumn{3}{c}{S_\text{max}}
\end{array}
\quad \text{or} \quad
\begin{array}{|c|c|c|}
\hline
\bullet & \phantom{\bullet} & \phantom{\bullet} \\ \hline
\bullet & \phantom{\bullet} & \phantom{\bullet} \\
\hline
\multicolumn{3}{c}{L_\text{max}}
\end{array} \quad .
\end{equation*}
Here, we prefer to start from the former one (maximum spin), which is naturally
suggested from the famous Hund's rule. It states that the multiplet
with the maximum $S$ has the lowest energy, from which we start our multiplet construction.

\section{Seniority}
We are so far lucky enough to diagonalize the Hamiltonian using ladder operators.
Recall how we constructed the multiplets for a $p^2$ orbital. We always start from
the largest $M_S$ and the corresponding largest $M_L$:
\begin{equation*}
\begin{matrix}
0 & 1 & 0 \\
1 & 2 & \boxed{1} \\
1 & 3 & 1 \\
1 & 2 & 1 \\
0 & 1 & 0
\end{matrix}
\quad \rightarrow \quad
\begin{matrix}
0 & \boxed{1} & 0 \\
0 & 1 & 0 \\
0 & 2 & 0 \\
0 & 1 & 0 \\
0 & 1 & 0
\end{matrix}
\quad \rightarrow \quad
\begin{matrix}
0 & 0 & 0 \\
0 & 0 & 0 \\
0 & \boxed{1} & 0 \\
0 & 0 & 0 \\
0 & 0 & 0
\end{matrix}
\end{equation*}
%
We were lucky because every time we started from a ``1''. This single vector
can be determined uniquely (by the basis vector itself or by the orthogonality relation).
But are we guaranteed to always start from a ``1''?
No, we are not. For example, the $M_L$-$M_S$ tables for the $d^3$ orbital:
\begin{equation*}
\begin{matrix}
0 & 1 & 1 & 0 \\ 
0 & 2 & 2 & 0 \\ 
1 & 4 & 4 & \boxed{1} \\ 
1 & 6 & 6 & 1 \\ 
2 & 8 & 8 & 2 \\ 
2 & 8 & 8 & 2 \\ 
2 & 8 & 8 & 2 \\ 
1 & 6 & 6 & 1 \\ 
1 & 4 & 4 & 1 \\ 
0 & 2 & 2 & 0 \\ 
0 & 1 & 1 & 0
\end{matrix}
\ \rightarrow \
\begin{matrix}
0 & 1 & 1 & 0 \\ 
0 & 2 & 2 & 0 \\ 
0 & 3 & 3 & 0 \\ 
0 & 5 & 5 & 0 \\ 
1 & 7 & 7 & \boxed{1} \\ 
1 & 7 & 7 & 1 \\ 
1 & 7 & 7 & 1 \\ 
0 & 5 & 5 & 0 \\ 
0 & 3 & 3 & 0 \\ 
0 & 2 & 2 & 0 \\ 
0 & 1 & 1 & 0
\end{matrix}
\ \rightarrow \
\begin{matrix}
0 & 1 & \boxed{1} & 0 \\ 
0 & 2 & 2 & 0 \\ 
0 & 3 & 3 & 0 \\ 
0 & 5 & 5 & 0 \\ 
0 & 6 & 6 & 0 \\ 
0 & 6 & 6 & 0 \\ 
0 & 6 & 6 & 0 \\ 
0 & 5 & 5 & 0 \\ 
0 & 3 & 3 & 0 \\ 
0 & 2 & 2 & 0 \\ 
0 & 1 & 1 & 0
\end{matrix}
\ \rightarrow \
\begin{matrix}
0 & 0 & 0 & 0 \\ 
0 & 1 & \boxed{1} & 0 \\ 
0 & 2 & 2 & 0 \\ 
0 & 4 & 4 & 0 \\ 
0 & 5 & 5 & 0 \\ 
0 & 5 & 5 & 0 \\ 
0 & 5 & 5 & 0 \\ 
0 & 4 & 4 & 0 \\ 
0 & 2 & 2 & 0 \\ 
0 & 1 & 1 & 0 \\ 
0 & 0 & 0 & 0
\end{matrix}
\ \rightarrow \
\begin{matrix}
0 & 0 & 0 & 0 \\ 
0 & 0 & 0 & 0 \\ 
0 & 1 & \boxed{1} & 0 \\ 
0 & 3 & 3 & 0 \\ 
0 & 4 & 4 & 0 \\ 
0 & 4 & 4 & 0 \\ 
0 & 4 & 4 & 0 \\ 
0 & 3 & 3 & 0 \\ 
0 & 1 & 1 & 0 \\ 
0 & 0 & 0 & 0 \\ 
0 & 0 & 0 & 0
\end{matrix}
\ \rightarrow \
\begin{matrix}
0 & 0 & 0 & 0 \\ 
0 & 0 & 0 & 0 \\ 
0 & 0 & 0 & 0 \\ 
0 & 2 & \boxed{2} & 0 \\ 
0 & 3 & 3 & 0 \\ 
0 & 3 & 3 & 0 \\ 
0 & 3 & 3 & 0 \\ 
0 & 2 & 2 & 0 \\ 
0 & 0 & 0 & 0 \\ 
0 & 0 & 0 & 0 \\ 
0 & 0 & 0 & 0
\end{matrix}
\end{equation*}
%
We encounter a ``2'', disaster! This ``2'' means there are two undetermined
eigen-vectors with $M_L=2$ and $M_S=1/2$. By the orthogonality relation,
we can at most find out an eigen-space (a plane) which is spanned by the two eigen-vectors.
But we cannot tell where exactly those two eigen-vectors are. That is the
limitation of using ladder operators. After all, we were doing something
peculiar: we diagonalized the matrix without using the matrix elements but only symmetries.
At this stage, we really have to ask help from the matrix elements. With the given
numerical values, we will be able to diagonalize the matrix completely.
But as we mentioned already, although we cannot obtain the eigen-vectors,
we can find the eigen-space spanned by these two eigen-vectors. To continue our work, we can generate two random
vectors and by applying the orthogonality relation, we get two random vectors
in the eigen-space. For example,
\begin{align}
|2,2,\frac{1}{2},\frac{1}{2}\rangle_1 = {} & \left( 0.211c_{2\downarrow}^\dagger c_{-1\uparrow}^\dagger c_{1\uparrow}^\dagger -0.259c_{1\downarrow}^\dagger c_{0\uparrow}^\dagger c_{1\uparrow}^\dagger -0.702c_{2\downarrow}^\dagger c_{-2\uparrow}^\dagger c_{2\uparrow}^\dagger \right. \nonumber \\
& \left. +0.490c_{1\downarrow}^\dagger c_{-1\uparrow}^\dagger c_{2\uparrow}^\dagger -0.279c_{0\downarrow}^\dagger c_{0\uparrow}^\dagger c_{2\uparrow}^\dagger +0.279c_{-1\downarrow}^\dagger c_{1\uparrow}^\dagger c_{2\uparrow}^\dagger \right)|0\rangle \label{eq:randv1} \\ 
|2,2,\frac{1}{2},\frac{1}{2}\rangle_2 = {} & \left( 0.382c_{2\downarrow}^\dagger c_{-1\uparrow}^\dagger c_{1\uparrow}^\dagger -0.468c_{1\downarrow}^\dagger c_{0\uparrow}^\dagger c_{1\uparrow}^\dagger -0.236c_{2\downarrow}^\dagger c_{-2\uparrow}^\dagger c_{2\uparrow}^\dagger \right. \nonumber \\
& \left. -0.146c_{1\downarrow}^\dagger c_{-1\uparrow}^\dagger c_{2\uparrow}^\dagger +0.528c_{0\downarrow}^\dagger c_{0\uparrow}^\dagger c_{2\uparrow}^\dagger -0.528c_{-1\downarrow}^\dagger c_{1\uparrow}^\dagger c_{2\uparrow}^\dagger \right)|0\rangle \label{eq:randv2}
\end{align}
%
Now apply both $L_-$ and $S_-$ to these two vectors, we obtain all other
vectors in this (2-dimensional) multiplet:
\begin{equation*}
\begin{matrix}
0 & 0 & 0 & 0 \\ 
0 & 0 & 0 & 0 \\ 
0 & 0 & 0 & 0 \\ 
0 & \boxed{2} & \boxed{2} & 0 \\ 
0 & \boxed{3} & \boxed{3} & 0 \\ 
0 & \boxed{3} & \boxed{3} & 0 \\ 
0 & \boxed{3} & \boxed{3} & 0 \\ 
0 & \boxed{2} & \boxed{2} & 0 \\ 
0 & 0 & 0 & 0 \\ 
0 & 0 & 0 & 0 \\ 
0 & 0 & 0 & 0
\end{matrix}
\quad \rightarrow \quad
\begin{matrix}
0 & 0 & 0 & 0 \\ 
0 & 0 & 0 & 0 \\ 
0 & 0 & 0 & 0 \\ 
0 & 0 & 0 & 0 \\ 
0 & 1 & 1 & 0 \\ 
0 & 1 & 1 & 0 \\ 
0 & 1 & 1 & 0 \\ 
0 & 0 & 0 & 0 \\ 
0 & 0 & 0 & 0 \\ 
0 & 0 & 0 & 0 \\ 
0 & 0 & 0 & 0
\end{matrix}
\end{equation*}
%
This step gives us two $^2D$ multiplets. We will denote it as $\stackrel{2\times}{^{2}D}$.
Once we encounter a multiplet which appears multiple times, it implies we could only find its
eigen-spaces but not eigen-vectors by using the ladder operator technique. We don't need to
worry about this problem for $s$ and $p$ shells. But for $d$ and higher shells, we will encounter
this problem quite often. A summary of atomic multiplets for open $s$, $p$, $d$ and $f$ shells
are tabulated in \ref{table:mtp}.

Recall the random vectors we generated in Eqns.~(\ref{eq:randv1}) and (\ref{eq:randv2}).
If we repeat the generation of random vectors, we might obtain a different set of vectors. They
span the same space but the vectors will have different directions. There is a very smart
method which allows us to construct those vectors uniquely (also with beautiful coefficients
like $1/\sqrt{3}$), although they are still not eigen-vectors.

This new concept is called seniority \cite{RIII}. The idea is that since we cannot uniquely determine
the vectors in the eigen-spaces of $\stackrel{2\times}{^{2}D}$ in $d^3$ shell, we start from
the $^2D$ in $d^1$ shell whose eigen-vectors can be uniquely determined.

\begin{center}
\begin{tabular}{l|l l l l l l l l}
\hline
 $d^1$ & & & $\boxed{^{2}D}$ & $\phantom{\stackrel{2\times}{^{2}D}}$ \\
 $d^2$ & $^{1}S$ & $^{3}P$ & $^{1}D$ & $^{3}F$ & $^{1}G$ & $\phantom{\stackrel{2\times}{^{2}D}}$ \\
 $d^3$ & & $^{2}P\,^{4}P$ & $\boxed{\stackrel{2\times}{^{2}D}}$ & $^{2}F\,^{4}F$ & $^{2}G$ & $^{2}H$ \\
\hline
\end{tabular}
\end{center}

The $d^1$ and $d^3$ shells are different by, of course, two electrons. To go from
$d^1$ to $d^3$, we need to add two more electrons. But the additional electrons
should not affect the angular momenta in $^2D$. Therefore, we should add electrons
according to the $^1S$ multiplet in $d^2$, which has $L=0$ and $S=0$.
This put-electron operation is called a seniority operation, one can think
the $^2D$ in $d^1$ as the parent and the $\stackrel{2\times}{^{2}D}$ in $d^3$ as the children.
The $^1S$ multiplet has one eigen-state
\begin{equation}
|0,0,0,0\rangle = \frac{1}{\sqrt{5}} \left( c_{2\downarrow}^\dagger c_{-2\uparrow}^\dagger -c_{1\downarrow}^\dagger c_{-1\uparrow}^\dagger +c_{0\downarrow}^\dagger c_{0\uparrow}^\dagger -c_{-1\downarrow}^\dagger c_{1\uparrow}^\dagger +c_{-2\downarrow}^\dagger c_{2\uparrow}^\dagger \right)|0\rangle
\end{equation}
%
The ``put-electron-operator'' (seniority operator) is simply
\begin{equation}
T = c_{2\downarrow}^\dagger c_{-2\uparrow}^\dagger -c_{1\downarrow}^\dagger c_{-1\uparrow}^\dagger +c_{0\downarrow}^\dagger c_{0\uparrow}^\dagger -c_{-1\downarrow}^\dagger c_{1\uparrow}^\dagger +c_{-2\downarrow}^\dagger c_{2\uparrow}^\dagger
\end{equation}
%
Apply this seniority operator to the leading vector of $^2D$ in $d^1$:
\begin{align}
T\, |2,2,\frac{1}{2},\frac{1}{2}\rangle & = T\, c_{2\uparrow}^\dagger |0\rangle \nonumber \\
& = \left( c_{2\downarrow}^\dagger c_{-2\uparrow}^\dagger -c_{1\downarrow}^\dagger c_{-1\uparrow}^\dagger +c_{0\downarrow}^\dagger c_{0\uparrow}^\dagger -c_{-1\downarrow}^\dagger c_{1\uparrow}^\dagger +c_{-2\downarrow}^\dagger c_{2\uparrow}^\dagger \right) c_{2\uparrow}^\dagger |0\rangle \nonumber \\
& = \left( c_{2\downarrow}^\dagger c_{-2\uparrow}^\dagger c_{2\uparrow}^\dagger -c_{1\downarrow}^\dagger c_{-1\uparrow}^\dagger c_{2\uparrow}^\dagger +c_{0\downarrow}^\dagger c_{0\uparrow}^\dagger c_{2\uparrow}^\dagger -c_{-1\downarrow}^\dagger c_{1\uparrow}^\dagger c_{2\uparrow}^\dagger \right) |0\rangle
\end{align}
%
We obtain the first leading vector of $^2D$ in $d^3$: (vectors should be normalized)
\begin{equation} \label{eq:snv1}
|2,2,\frac{1}{2},\frac{1}{2},0\rangle = \frac{1}{\sqrt{4}} \left( c_{2\downarrow}^\dagger c_{-2\uparrow}^\dagger c_{2\uparrow}^\dagger -c_{1\downarrow}^\dagger c_{-1\uparrow}^\dagger c_{2\uparrow}^\dagger +c_{0\downarrow}^\dagger c_{0\uparrow}^\dagger c_{2\uparrow}^\dagger -c_{-1\downarrow}^\dagger c_{1\uparrow}^\dagger c_{2\uparrow}^\dagger \right)|0\rangle
\end{equation}
%
The fifth index $W=0$ is called the seniority number. One can think it as
the index of an array.\footnote{A more standard way to assign the seniority number
is to name according to their parents. For example, our first vector is produced from
$^2D$ in $d^1$, hence it has a seniority number 1. The second vector is produced
in $d^3$ from the orthogonality relation, hence a seniority number 3. However,
this ancestry relation gets much more complicated for an $f$ shell. We would
like to assign the seniority number just as the array index. Actually
it doesn't matter how we name the seniority number since it does not really
have a physical meaning. We chose it for convenience as long as it can label
a vector uniquely.}
The remaining leading vector $|2,2,\frac{1}{2},\frac{1}{2},1\rangle$
can then be uniquely determined from the orthogonality relation.
in shell $d^3$.
\begin{align} \label{eq:snv2}
|2,2,\frac{1}{2},\frac{1}{2},1\rangle & = \frac{1}{\sqrt{84}} \left( 4c_{2\downarrow}^\dagger c_{-1\uparrow}^\dagger c_{1\uparrow}^\dagger -\sqrt{24}c_{1\downarrow}^\dagger c_{0\uparrow}^\dagger c_{1\uparrow}^\dagger -5c_{2\downarrow}^\dagger c_{-2\uparrow}^\dagger c_{2\uparrow}^\dagger \right. \nonumber \\
& \left. \qquad\qquad +c_{1\downarrow}^\dagger c_{-1\uparrow}^\dagger c_{2\uparrow}^\dagger +3c_{0\downarrow}^\dagger c_{0\uparrow}^\dagger c_{2\uparrow}^\dagger -3c_{-1\downarrow}^\dagger c_{1\uparrow}^\dagger c_{2\uparrow}^\dagger \right)|0\rangle
\end{align}
%
The vectors in Eqns.~(\ref{eq:snv1}) and (\ref{eq:snv2}) span the same space
as the vectors in Eqns.~(\ref{eq:randv1}) and (\ref{eq:randv2}). Now those vectors
can be uniquely determined and addressed by five quantum numbers $\Ket{L,M_L,S,M_S,W}$.
By applying ladder operators to these two leading vectors,
we obtain the complete set of vectors in the multiplet $\stackrel{2\times}{^2D}$.
One must be alerted that $\Ket{L,M_L,S,M_S,W}$ is, in general, not an eigen-vector.
But the complete set $\Ket{L,M_L,S,M_S,0}\ldots\, \Ket{L,M_L,S,M_S,n-1}$ spans the complete
eigen-space of multiplets $\stackrel{\qquad n\times}{^{2S+1}L}$.

\clearpage
\begin{table}[h!]
\begin{center}
\small
\rotatebox{-90}{
\begin{minipage}{\textheight}
\caption{Atomic multiplets for open $s$, $p$, $d$ and $f$ shells.
All closed shells have $^1S$.}
\label{table:mtp}
\makebox[1.05\textheight]{
\begin{tabular}{l|l l l l l l l l l l l l l}
\hline
 Orbital & \multicolumn{2}{l}{Multiplets} \\
\hline\hline
 $s^1$ & $^{2}S$ & $\phantom{\stackrel{2\times}{^{2}D}}$ \\
\hline
 $p^1$ $p^5$ & & $^{2}P$ & $\phantom{\stackrel{2\times}{^{2}D}}$ \\
 $p^2$ $p^4$ & $^{1}S$ & $^{3}P$ & $^{1}D$ & $\phantom{\stackrel{2\times}{^{2}D}}$ \\
 $p^3$ & $^{4}S$ & $^{2}P$ & $^{2}D$ & $\phantom{\stackrel{2\times}{^{2}D}}$ \\
\hline
 $d^1$ $d^9$ & & & $^{2}D$ & $\phantom{\stackrel{2\times}{^{2}D}}$ \\
 $d^2$ $d^8$ & $^{1}S$ & $^{3}P$ & $^{1}D$ & $^{3}F$ & $^{1}G$ & $\phantom{\stackrel{2\times}{^{2}D}}$ \\
 $d^3$ $d^7$ & & $^{2}P\,^{4}P$ & $\stackrel{2\times}{^{2}D}$ & $^{2}F\,^{4}F$ & $^{2}G$ & $^{2}H$ \\
 $d^4$ $d^6$ & $\stackrel{2\times}{^{1}S}$ & $\stackrel{2\times}{^{3}P}$ & $\stackrel{2\times}{^{1}D} {^{3}D}\,^{5}D$ & ${^{1}F} \stackrel{2\times}{^{3}F}$ & $\stackrel{2\times}{^{1}G} {^{3}G}$ & $^{3}H$ & $^{1}I$ \\
 $d^5$ & $^{2}S\,^{6}S$ & $^{2}P\,^{4}P$ & $\stackrel{3\times}{^{2}D} {^{4}D}$ & $\stackrel{2\times}{^{2}F} {^{4}F}$ & $\stackrel{2\times}{^{2}G} {^{4}G}$ & $^{2}H$ & $^{2}I$ \\
\hline
 $f^1$ $f^{13}$ & & & & $^{2}F$ & $\phantom{\stackrel{2\times}{^{2}D}}$ \\
 $f^2$ $f^{12}$ & $^{1}S$ & $^{3}P$ & $^{1}D$ & $^{3}F$ & $^{1}G$ & $^{3}H$ & $^{1}I$ & $\phantom{\stackrel{2\times}{^{2}D}}$ \\
 $f^3$ $f^{11}$ & $^{4}S$ & $^{2}P$ & $\stackrel{2\times}{^{2}D} {^{4}D}$ & $\stackrel{2\times}{^{2}F} {^{4}F}$ & $\stackrel{2\times}{^{2}G} {^{4}G}$ & $\stackrel{2\times}{^{2}H}$ & $^{2}I\,^{4}I$ & $^{2}K$ & $^{2}L$ \\
 $f^4$ $f^{10}$ & $\stackrel{2\times}{^{1}S} {^{5}S}$ & $\stackrel{3\times}{^{3}P}$ & $\stackrel{4\times}{^{1}D}\,\stackrel{2\times}{^{3}D} {^{5}D}$ & ${^{1}F} \stackrel{4\times}{^{3}F} {^{5}F}$ & $\stackrel{4\times}{^{1}G}\,\stackrel{3\times}{^{3}G} {^{5}G}$ & $\stackrel{2\times}{^{1}H}\,\stackrel{4\times}{^{3}H}$ & $\stackrel{3\times}{^{1}I}\,\stackrel{2\times}{^{3}I} {^{5}I}$ & ${^{1}K} \stackrel{2\times}{^{3}K}$ & $\stackrel{2\times}{^{1}L} {^{3}L}$ & $^{3}M$ & $^{1}N$ \\
 $f^5$ $f^9$ & $^{4}S$ & $\stackrel{4\times}{^{2}P}\,\stackrel{2\times}{^{4}P} {^{6}P}$ & $\stackrel{5\times}{^{2}D}\,\stackrel{3\times}{^{4}D}$ & $\stackrel{7\times}{^{2}F}\,\stackrel{4\times}{^{4}F} {^{6}F}$ & $\stackrel{6\times}{^{2}G}\,\stackrel{4\times}{^{4}G}$ & $\stackrel{7\times}{^{2}H}\,\stackrel{3\times}{^{4}H} {^{6}H}$ & $\stackrel{5\times}{^{2}I}\,\stackrel{3\times}{^{4}I}$ & $\stackrel{5\times}{^{2}K}\,\stackrel{2\times}{^{4}K}$ & $\stackrel{3\times}{^{2}L} {^{4}L}$ & $\stackrel{2\times}{^{2}M} {^{4}M}$ & $^{2}N$ & $^{2}O$ \\
 $f^6$ $f^8$ & $\stackrel{4\times}{^{1}S} {^{5}S}$ & ${^{1}P} \stackrel{6\times}{^{3}P} {^{5}P}$ & $\stackrel{6\times}{^{1}D}\,\stackrel{5\times}{^{3}D}\,\stackrel{3\times}{^{5}D}$ & $\stackrel{4\times}{^{1}F}\,\stackrel{9\times}{^{3}F}\,\stackrel{2\times}{^{5}F} {^{7}F}$ & $\stackrel{8\times}{^{1}G}\,\stackrel{7\times}{^{3}G}\,\stackrel{3\times}{^{5}G}$ & $\stackrel{4\times}{^{1}H}\,\stackrel{9\times}{^{3}H}\,\stackrel{2\times}{^{5}H}$ & $\stackrel{7\times}{^{1}I}\,\stackrel{6\times}{^{3}I}\,\stackrel{2\times}{^{5}I}$ & $\stackrel{3\times}{^{1}K}\,\stackrel{6\times}{^{3}K} {^{5}K}$ & $\stackrel{4\times}{^{1}L}\,\stackrel{3\times}{^{3}L} {^{5}L}$ & $\stackrel{2\times}{^{1}M}\,\stackrel{3\times}{^{3}M}$ & $\stackrel{2\times}{^{1}N} {^{3}N}$ & $^{3}O$ & $^{1}Q$ \\
 $f^7$ & $\stackrel{2\times}{^{2}S}\,\stackrel{2\times}{^{4}S} {^{8}S}$ & $\stackrel{5\times}{^{2}P}\,\stackrel{2\times}{^{4}P} {^{6}P}$ & $\stackrel{7\times}{^{2}D}\,\stackrel{6\times}{^{4}D} {^{6}D}$ & $\stackrel{10\times}{^{2}F}\,\stackrel{5\times}{^{4}F} {^{6}F}$ & $\stackrel{10\times}{^{2}G}\,\stackrel{7\times}{^{4}G} {^{6}G}$ & $\stackrel{9\times}{^{2}H}\,\stackrel{5\times}{^{4}H} {^{6}H}$ & $\stackrel{9\times}{^{2}I}\,\stackrel{5\times}{^{4}I} {^{6}I}$ & $\stackrel{7\times}{^{2}K}\,\stackrel{3\times}{^{4}K}$ & $\stackrel{5\times}{^{2}L}\,\stackrel{3\times}{^{4}L}$ & $\stackrel{4\times}{^{2}M} {^{4}M}$ & $\stackrel{2\times}{^{2}N} {^{4}N}$ & $^{2}O$ & $^{2}Q$ \\
\hline
\end{tabular}
}
\end{minipage}
}
\end{center}
\end{table}
\clearpage

\section{Eigen-energy of multiplet states}
As we emphasized earlier, the eigen-energies of the Hamiltonian could be
obtained by throwing the matrix into a diagonalization solver. However, from this
``brute-force'' approach, it cannot be seen that which eigen-energy corresponds to which
multiplet. Now, since we have constructed the eigen-vectors already (up to seniority),
we can easily compute the eigen-energies by matrix-vector multiplications.
For a given eigen-vector $\Ket{\vec{v}_n}$, the corresponding eigen-energy is simply
\begin{equation}\label{eq:En}
E_n = \Bra{\vec{v}_n} H_U \Ket{\vec{v}_n}
\end{equation}
%
The Hamiltonian consists of two major parts: the Slater-Condon
parameters and the Gaunt coefficients. For on-shell interactions, the Slater-Condon
parameters enter as pre-factors $F^{(k)}$ since they depend on only $n$ and $l$ (see
Eqn.~(\ref{eq:Fk})). It is more convenient to take out those pre-factors
and to write our Hamiltonian as,
\begin{equation}\label{eq:Hk}
H_U = \sum_{k=0(+2)}^{2l} F^{(k)} \tilde{H}_U^{(k)}
\end{equation}
where $\tilde{H}_U^{(k)}$ has only a dependence on the Gaunt coefficients,
which are system independent.
For a given shell, the sum rule in Eqn.~(\ref{eq:kRule}) requires $k=0,2,\ldots,2l$.
The routine for setting up $\tilde{H}_U^{(k)}$ is identical to Algorithm~\ref{alg:Huij},
except that one needs to replace $U_{\alpha\beta\gamma\delta}$ by
$\delta_{\sigma_1\sigma_4} \delta_{\sigma_2\sigma_3} \frac{4\pi}{2k+1} A^{(k)}(lm_1,lm_2,lm_3,lm_4)$.
The eigenvalues $\tilde{E}_n^{(k)}$ from $\tilde{H}_U^{(k)}$ are universal for all atoms.
We can easily compute $\tilde{E}_n^{(k)}$ by applying our eigen-vectors to $\tilde{H}_U^{(k)}$:
\begin{equation}\label{eq:Ekn}
\tilde{E}_n^{(k)} = \Bra{\vec{v}_n} \tilde{H}_U^{(k)} \Ket{\vec{v}_n}
\end{equation}
%
It is not trivial that the eigen-vectors of $H_U$ are simultaneously eigen-vectors of
$\tilde{H}_U^{(k)}$. This is because $\tilde{H}_U^{(k)}$ also commute with $\vec{L}$
and $\vec{S}$ (up to seniority). Now, system independently, we can write the eigen-energies
as (think $F^{(k)}$ as pre-factors)
\begin{equation}\label{eq:Eksum}
E_n = \sum_{k=0(+2)}^{2l} F^{(k)} \tilde{E}_n^{(k)}
\end{equation}
%
Those pre-factors $F^{(k)}$ are subject to the input wave functions which
are determined by different atoms, electronic
configurations and approximation methods (e.g. self-consistent field approximation).
Table~\ref{table:eig} summarizes the multiplet eigen-energies for a few selected
atomic shells. The terms are sorted in the descending order of multiplicity
$2S+1$. For a given multiplicity, terms are sorted in the descending order
of $L$. The first term of each individual table is the multiplet with the
lowest energy (Hund's rule).

To get a direct impression of how Table~\ref{table:eig} works, we take the $p^2$
configuration as an example. A carbon atom with electronic configuration
$1s^2$ $2s^2$ $2p^2$ would be a good candidate for this demonstration.
From the self-consistent calculation, we obtained the eigen-energy of the
$2p$ orbital to be $-0.199186$ Hartree (see Table~\ref{table:SCFeigE}).
From the resulting wave function $u_{2p}$, we can easily compute the
Slater-Condon parameters using Eqn.~(\ref{eq:Fk}),
\begin{align} \label{eq:p2Fk}
\begin{split}
F^{(0)}(2p) & = 0.520216\ (\text{Hartree}) \\
F^{(2)}(2p) & = 0.229662\ (\text{Hartree})
\end{split}
\end{align}
Putting (\ref{eq:p2Fk}) into Table~\ref{table:eig}, we obtain the multiplet energies,
\begin{align} \label{eq:p2mtpE}
\begin{split}
^3P: &\ 0.474284\ (\text{Hartree}) \\
^1D: &\ 0.529402\ (\text{Hartree}) \\
^1S: &\ 0.612081\ (\text{Hartree})
\end{split}
\end{align}
%
These are the three split eigen-energies of the $2p^2$ configuration.\footnote{We
limited our basis within the open shell. This gives reasonably good approximations since
multiplet splittings are normally much smaller than the energy differences among electronic
shells (first order perturbation theory).}
It is important to understand that they are the eigen-energies
from the Coulomb repulsion Hamiltonian
\begin{equation}
H_U = \sum_{i<j}^N \frac{1}{|\vec{r}_i - \vec{r}_j|}
\end{equation}
But not the full Hamiltonian
\begin{equation}
H = \sum_{i=1}^N \left[ -\frac{1}{2} \nabla_i^2 - \frac{Z}{r_i} \right] + \sum_{i<j}^N \frac{1}{|\vec{r}_i - \vec{r}_j|}
\end{equation}
%
The eigen-energies in (\ref{eq:p2mtpE}) are not the absolute energies.
They are correct up to a constant shift. Nevertheless, their energy splittings
are normally the quantities that we are interested in.
\begin{figure}[h!]
\begin{center}
\begin{tikzpicture}[scale=0.175]
\draw[very thick] (0,13.7797) -- (15,13.7797);
\draw[very thick] (40,0) -- (55,0);
\draw[very thick] (40,11.0236) -- (55,11.0236);
\draw[very thick] (40,27.5594) -- (55,27.5594);
%
\node at (8,15.7797) {$p^2$};
\node at (47,2) {$^3P$};
\node at (47,13.0236) {$^1D$};
\node at (47,29.5594) {$^1S$};
%
\draw[very thin, gray, dashed] (15,13.7797) -- (40,0);
\draw[very thin, gray, dashed] (15,13.7797) -- (40,11.0236);
\draw[very thin, gray, dashed] (15,13.7797) -- (40,27.5594);
%
\draw[triangle 45-triangle 45] (52,0) -- (52,11.0236);
\draw[triangle 45-triangle 45] (52,11.0236) -- (52,27.5594);
%
\node at (60,5.5118) {$\Delta E = 0.055118$};
\node at (60,19.2915) {$\Delta E = 0.082679$};
\end{tikzpicture}
\end{center}
\caption{Energy splitting of the $p^2$ configuration of a carbon atom.
Energies are given in units of Hartree (a.u.).}
\label{fig:p2split}
\end{figure}

We should keep in mind that we didn't obtain all eigen-vectors from the ladder operator
technique. For the case where the seniority numbers are required,
we can at most construct a small eigen-space that the eigen-vectors live in.
From this eigen-space, namely, a set of vectors $\Ket{\vec{v}_i}$, we compute
the matrix
\begin{equation}\label{eq:Ekij}
\tilde{E}_{ij}^{(k)} = \Bra{\vec{v}_i} \tilde{H}_U^{(k)} \Ket{\vec{v}_j}
\end{equation}
which contains the eigen-energies. Similar to Table~\ref{table:eig}, we also provide
the eigen-energies of multiplets with seniority in Table~\ref{table:eigm}.
But instead of numbers being coefficients, we have small matrices as the coefficients.
For a given set of $F^{(k)}$, we can sum up the small matrices and diagonalize
to obtain the eigen-energies numerically.

\clearpage
\begin{table}
\caption{Multiplet (without seniority cases) eigen-energies for a few selected atomic shells.}
\label{table:eig}
\vspace{1em}
\begin{minipage}{0.5\textwidth}
\begin{equation*}
\renewcommand{\arraystretch}{1.8}
\begin{array}{|>{\displaystyle}c|>{\displaystyle}c|>{\displaystyle}r >{\displaystyle}c >{\displaystyle}r|}
\hline
p^2 & ^3P & F^{(0)} & - & \frac{1}{5} F^{(2)} \\
    & ^1D & F^{(0)} & + & \frac{1}{25} F^{(2)} \\
    & ^1S & F^{(0)} & + & \frac{2}{5} F^{(2)} \\[0.5em]
\hline
\end{array}
\end{equation*}
\begin{equation*}
\renewcommand{\arraystretch}{1.8}
\begin{array}{|>{\displaystyle}c|>{\displaystyle}c|>{\displaystyle}r >{\displaystyle}c >{\displaystyle}r|}
\hline
p^3 & ^4S & 3 F^{(0)} & - & \frac{3}{5} F^{(2)} \\
    & ^2D & 3 F^{(0)} & - & \frac{6}{25} F^{(2)} \\
    & ^2P & 3 F^{(0)} &   &  \\[0.5em]
\hline
\end{array}
\end{equation*}
\begin{equation*}
\renewcommand{\arraystretch}{1.8}
\begin{array}{|>{\displaystyle}c|>{\displaystyle}c|>{\displaystyle}r >{\displaystyle}c >{\displaystyle}r >{\displaystyle}c >{\displaystyle}r|}
\hline
d^2 & ^3F & F^{(0)} & - & \frac{8}{49} F^{(2)} & - & \frac{1}{49} F^{(4)} \\
    & ^3P & F^{(0)} & + & \frac{1}{7} F^{(2)} & - & \frac{4}{21} F^{(4)} \\
    & ^1G & F^{(0)} & + & \frac{4}{49} F^{(2)} & + & \frac{1}{441} F^{(4)} \\
    & ^1D & F^{(0)} & - & \frac{3}{49} F^{(2)} & + & \frac{4}{49} F^{(4)} \\
    & ^1S & F^{(0)} & + & \frac{2}{7} F^{(2)} & + & \frac{2}{7} F^{(4)} \\[0.5em]
\hline
\end{array}
\end{equation*}
\end{minipage}
\begin{minipage}{0.5\textwidth}
\begin{equation*}
\renewcommand{\arraystretch}{1.8}
\begin{array}{|>{\displaystyle}c|>{\displaystyle}c|>{\displaystyle}r >{\displaystyle}c >{\displaystyle}r >{\displaystyle}c >{\displaystyle}r|}
\hline
d^5 & ^6S & 10 F^{(0)} & - & \frac{5}{7} F^{(2)} & - & \frac{5}{7} F^{(4)} \\
    & ^4G & 10 F^{(0)} & - & \frac{25}{49} F^{(2)} & - & \frac{190}{441} F^{(4)} \\
    & ^4F & 10 F^{(0)} & - & \frac{13}{49} F^{(2)} & - & \frac{20}{49} F^{(4)} \\
    & ^4D & 10 F^{(0)} & - & \frac{18}{49} F^{(2)} & - & \frac{25}{49} F^{(4)} \\
    & ^4P & 10 F^{(0)} & - & \frac{4}{7} F^{(2)} & - & \frac{5}{21} F^{(4)} \\
    & ^2I & 10 F^{(0)} & - & \frac{24}{49} F^{(2)} & - & \frac{10}{49} F^{(4)} \\
    & ^2H & 10 F^{(0)} & - & \frac{22}{49} F^{(2)} & - & \frac{10}{147} F^{(4)} \\
    & ^2P & 10 F^{(0)} & + & \frac{20}{49} F^{(2)} & - & \frac{80}{147} F^{(4)} \\
    & ^2S & 10 F^{(0)} & - & \frac{3}{49} F^{(2)} & - & \frac{65}{147} F^{(4)} \\[0.5em]
\hline
\end{array}
\end{equation*}
\end{minipage}

\begin{minipage}{0.5\textwidth}
\begin{equation*}
\renewcommand{\arraystretch}{1.8}
\begin{array}{|>{\displaystyle}c|>{\displaystyle}c|>{\displaystyle}r >{\displaystyle}c >{\displaystyle}r >{\displaystyle}c >{\displaystyle}r|}
\hline
d^3 & ^4F & 3 F^{(0)} & - & \frac{15}{49} F^{(2)} & - & \frac{8}{49} F^{(4)} \\
    & ^4P & 3 F^{(0)} &   &   & - & \frac{1}{3} F^{(4)} \\
    & ^2H & 3 F^{(0)} & - & \frac{6}{49} F^{(2)} & - & \frac{4}{147} F^{(4)} \\
    & ^2G & 3 F^{(0)} & - & \frac{11}{49} F^{(2)} & + & \frac{13}{441} F^{(4)} \\
    & ^2F & 3 F^{(0)} & + & \frac{9}{49} F^{(2)} & - & \frac{29}{147} F^{(4)} \\
    & ^2P & 3 F^{(0)} & - & \frac{6}{49} F^{(2)} & - & \frac{4}{147} F^{(4)} \\[0.5em]
\hline
\end{array}
\end{equation*}
\end{minipage}
\begin{minipage}{0.5\textwidth}
\begin{equation*}
\renewcommand{\arraystretch}{1.8}
\begin{array}{|>{\displaystyle}c|>{\displaystyle}c|>{\displaystyle}r >{\displaystyle}c >{\displaystyle}r >{\displaystyle}c >{\displaystyle}r|}
\hline
d^4 & ^5D & 6 F^{(0)} & - & \frac{3}{7} F^{(2)} & - & \frac{3}{7} F^{(4)} \\
    & ^3H & 6 F^{(0)} & - & \frac{17}{49} F^{(2)} & - & \frac{23}{147} F^{(4)} \\
    & ^3G & 6 F^{(0)} & - & \frac{12}{49} F^{(2)} & - & \frac{94}{441} F^{(4)} \\
    & ^3D & 6 F^{(0)} & - & \frac{5}{49} F^{(2)} & - & \frac{43}{147} F^{(4)} \\
    & ^1I & 6 F^{(0)} & - & \frac{15}{49} F^{(2)} & - & \frac{1}{49} F^{(4)} \\
    & ^1F & 6 F^{(0)} &   &   & - & \frac{4}{21} F^{(4)} \\[0.5em]
\hline
\end{array}
\end{equation*}
\end{minipage}
\end{table}
\clearpage
\begin{table}
\captionof{table}{Multiplet (seniority cases) eigen-energies for a few selected atomic shells.}
\label{table:eigm}
\begin{equation*}
\begin{array}{|>{\displaystyle}c|>{\displaystyle}c|>{\displaystyle}r >{\displaystyle}c >{\displaystyle}r >{\displaystyle}c >{\displaystyle}r|}
\hline
d^3 & \stackrel{2\times}{^2D} & \begin{bmatrix}3 & 0 \\ 0 & 3\end{bmatrix} F^{(0)} & - & \begin{bmatrix}-\frac{1}{7} & \frac{3\sqrt{21}}{49} \\ \frac{3\sqrt{21}}{49} & -\frac{3}{49}\end{bmatrix} F^{(2)} & + & \begin{bmatrix}\frac{1}{7} & \frac{5\sqrt{21}}{147} \\ \frac{5\sqrt{21}}{147} & -\frac{19}{147}\end{bmatrix} F^{(4)} \\
\hline
\end{array}
\end{equation*}

\begin{equation*}
\begin{array}{|>{\displaystyle}c|>{\displaystyle}c|>{\displaystyle}r >{\displaystyle}c >{\displaystyle}r >{\displaystyle}c >{\displaystyle}r|}
\hline
d^4 & \stackrel{2\times}{^3F} & \begin{bmatrix}6 & 0 \\ 0 & 6\end{bmatrix} F^{(0)} & - & \begin{bmatrix}\phantom{0}\frac{2}{49}\phantom{0} & \phantom{0}\frac{12}{49}\phantom{0} \\ \frac{12}{49} & \frac{8}{49}\end{bmatrix} F^{(2)} & + & \begin{bmatrix}\,-\frac{13}{147} & \frac{20}{147} \\ \,\frac{20}{147} & -\frac{38}{147}\phantom{,}\end{bmatrix} F^{(4)} \\[1.2em]
    & \stackrel{2\times}{^3P} & \begin{bmatrix}6 & 0 \\ 0 & 6\end{bmatrix} F^{(0)} & + & \begin{bmatrix}-\frac{1}{7} & \frac{4\sqrt{14}}{49} \\ \frac{4\sqrt{14}}{49} & -\frac{3}{49}\end{bmatrix} F^{(2)} & - & \begin{bmatrix}\frac{2}{63} & \frac{20\sqrt{14}}{441} \\ \frac{20\sqrt{14}}{441} & \frac{139}{441}\end{bmatrix} F^{(4)} \\[1.2em]
    & \stackrel{2\times}{^1G} & \begin{bmatrix}6 & 0 \\ 0 & 6\end{bmatrix} F^{(0)} & - & \begin{bmatrix}\frac{6}{49} & \frac{4\sqrt{11}}{49} \\ \frac{4\sqrt{11}}{49} & \frac{4}{49}\end{bmatrix} F^{(2)} & + & \begin{bmatrix}\frac{17}{147} & \frac{20\sqrt{11}}{441} \\ \frac{20\sqrt{11}}{441} & -\frac{64}{441}\end{bmatrix} F^{(4)} \\[1.2em]
    & \stackrel{2\times}{^1D} & \begin{bmatrix}6 & 0 \\ 0 & 6\end{bmatrix} F^{(0)} & + & \begin{bmatrix}\frac{15}{49} & \frac{12\sqrt{2}}{49} \\ \frac{12\sqrt{2}}{49} & \frac{3}{49}\end{bmatrix} F^{(2)} & - & \begin{bmatrix}\phantom{,}\frac{6}{49} & \,\frac{20\sqrt{2}}{147}\phantom{,} \\ \phantom{,}\frac{20\sqrt{2}}{147} & \frac{11}{49}\end{bmatrix} F^{(4)} \\[1.2em]
    & \stackrel{2\times}{^1S} & \begin{bmatrix}6 & 0 \\ 0 & 6\end{bmatrix} F^{(0)} & - & \begin{bmatrix}-\frac{2}{7} & \frac{6\sqrt{21}}{49} \\ \frac{6\sqrt{21}}{49} & -\frac{6}{49}\end{bmatrix} F^{(2)} & + & \begin{bmatrix}\frac{2}{7} & \frac{10\sqrt{21}}{147} \\ \frac{10\sqrt{21}}{147} & -\frac{38}{147}\end{bmatrix} F^{(4)} \\
\hline
\end{array}
\end{equation*}

\begin{equation*}
\begin{array}{|>{\displaystyle}c|>{\displaystyle}c|>{\displaystyle}r >{\displaystyle}c >{\displaystyle}r >{\displaystyle}c >{\displaystyle}r|}
\hline
d^5 & \stackrel{2\times}{^2G} & \begin{bmatrix}10 & 0 \\ 0 & 10\end{bmatrix} F^{(0)} & + & \begin{bmatrix}\frac{3}{49} & 0 \\ 0 & -\frac{13}{49}\end{bmatrix} F^{(2)} & - & \begin{bmatrix}\frac{155}{441} & 0 \\ 0 & \frac{145}{441}\end{bmatrix} F^{(4)} \\[1.2em]
    & \stackrel{2\times}{^2F} & \begin{bmatrix}10 & 0 \\ 0 & 10\end{bmatrix} F^{(0)} & - & \begin{bmatrix}\frac{25}{49} & 0 \\ 0 & \phantom{0}\frac{9}{49}\phantom{,}\end{bmatrix} F^{(2)} & - & \begin{bmatrix}\frac{5}{147} & 0 \\ 0 & \frac{55}{147}\end{bmatrix} F^{(4)} \\
\hline
\end{array}
\end{equation*}

\begin{equation*}
\begin{array}{|>{\displaystyle}c|>{\displaystyle}c|>{\displaystyle}r >{\displaystyle}c >{\displaystyle}r >{\displaystyle}c >{\displaystyle}r|}
\hline
d^5 & \stackrel{3\times}{^2D} & \begin{bmatrix}10 & 0 & 0 \\ 0 & 10 & 0 \\ 0 & 0 & 10\end{bmatrix} F^{(0)} & - & \begin{bmatrix}0 & 0 & \frac{6\sqrt{14}}{49} \\ 0 & \frac{4}{49} & 0 \\ \frac{6\sqrt{14}}{49} & 0 & \frac{6}{49}\end{bmatrix} F^{(2)} & + & \begin{bmatrix}0 & 0 & \frac{10\sqrt{14}}{147} \\ 0 & -\frac{40}{147} & 0 \\ \frac{10\sqrt{14}}{147} & 0 & -\frac{20}{49}\end{bmatrix} F^{(4)} \\
\hline
\end{array}
\end{equation*}
\end{table}
\clearpage

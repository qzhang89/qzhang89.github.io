\chapter{Second quantization} \label{app:B}

\section{A different formalism but the same physics}
In real space, electrons are described as wave functions.
Because electrons are fermions, their anti-symmetric wave functions are
formulated as Slater determinants \cite{GBK}. (as a remark, a many-electron wave function
$\Psi(\vec{r}_1,\ldots,\vec{r}_N)$ is in general a linear combination
of Slater determinants. Meanwhile, we are ignoring spin to simplify our notations.)
\begin{equation} \label{eq:slaterdeterm}
\Phi_{\alpha_1\cdots\alpha_N}(\vec{r}_1,\ldots,\vec{r}_N) = \frac{1}{\sqrt{N!}}
\begin{vmatrix}
\varphi_{\alpha_1}(\vec{r}_1) & \varphi_{\alpha_2}(\vec{r}_1) & \cdots & \varphi_{\alpha_N}(\vec{r}_1) \\
\varphi_{\alpha_1}(\vec{r}_2) & \varphi_{\alpha_2}(\vec{r}_2) & \cdots & \varphi_{\alpha_N}(\vec{r}_2) \\
\vdots & \vdots & \ddots & \vdots \\
\varphi_{\alpha_1}(\vec{r}_N) & \varphi_{\alpha_2}(\vec{r}_N) & \cdots & \varphi_{\alpha_N}(\vec{r}_N)
\end{vmatrix}
\end{equation}
%
For a one-electron wave function, Eqn.~(\ref{eq:slaterdeterm}) is trivial:
\begin{equation}
\Phi_\alpha(\vec{r}_1) = \varphi_\alpha(\vec{r}_1)
\end{equation}
%
For a two-electron wave function, Eqn.~(\ref{eq:slaterdeterm}) reads:
\begin{equation}
\Phi_{\alpha\beta}(\vec{r}_1,\vec{r}_2) = \frac{1}{\sqrt{2}}(\varphi_\alpha(\vec{r}_1)\varphi_\beta(\vec{r}_2)
- \varphi_\beta(\vec{r}_1)\varphi_\alpha(\vec{r}_2))
\end{equation}
%
For a three-electron wave function, Eqn.~(\ref{eq:slaterdeterm}) becomes:
\begin{align} \label{eq:3SD}
\Phi_{\alpha\beta\gamma}(\vec{r}_1,\vec{r}_2,\vec{r}_3) = & {} \frac{1}{\sqrt{6}}
\Big(\phantom{-}\varphi_\alpha(\vec{r}_1)\varphi_\beta(\vec{r}_2)\varphi_\gamma(\vec{r}_3)
+ \varphi_\gamma(\vec{r}_1)\varphi_\alpha(\vec{r}_2)\varphi_\beta(\vec{r}_3)
+ \varphi_\beta(\vec{r}_1)\varphi_\gamma(\vec{r}_2)\varphi_\alpha(\vec{r}_3) \nonumber \\
& {} \phantom{\frac{1}{\sqrt{6}}(} {-\varphi_\gamma(\vec{r}_1)}\varphi_\beta(\vec{r}_2)\varphi_\alpha(\vec{r}_3)
- \varphi_\beta(\vec{r}_1)\varphi_\alpha(\vec{r}_2)\varphi_\gamma(\vec{r}_3)
- \varphi_\alpha(\vec{r}_1)\varphi_\gamma(\vec{r}_2)\varphi_\beta(\vec{r}_3) \Big)
\end{align}
%
I wouldn't intend to write the four-electron wave function since it will be super long.
This trouble is actually caused by working with real space.
Since we need to label $\vec{r}_1, \vec{r}_2, \ldots, \vec{r}_N$ for different
degrees of freedom, we must use the Slater determinant to ensure the anti-symmetry
property of the wave function, which unfortunately makes the expression very complicated.
We can get rid of this difficulty if not working with real space. To specify an electron
in state $\alpha$, instead of $\varphi_\alpha(\vec{r}_1)$, we write
\begin{equation*}
\Ket{\alpha}
\end{equation*}
%
which is known as the Dirac state \cite{QM}. Now, for a two-electron state, we write
\begin{equation*}
\Ket{\alpha, \beta}
\end{equation*}
%
But how do we ensure the anti-symmetry of this two-electron state?
\begin{equation}
\Ket{\alpha, \beta} = -\Ket{\beta, \alpha}
\end{equation}
%
Previously, when working with real space, this was ensured by the
Slater determinant. Now, this anti-symmetry will be taken care by the
second quantization operators.\footnote{The order of the operators indicates
we create $\alpha$ first and $\beta$ second. Hence we write
$c_\beta^\dagger c_\alpha^\dagger \Ket{0} = \Ket{\alpha,\beta}$.}
\begin{equation}
\Ket{\alpha, \beta} = c_\beta^\dagger c_\alpha^\dagger \Ket{0}
\end{equation}
%
These lovely operators have the property that if they change order, they produce
a minus sign (the fermi-sign).
\begin{equation}
\Ket{\alpha, \beta} = c_\beta^\dagger c_\alpha^\dagger \Ket{0} =
- c_\alpha^\dagger c_\beta^\dagger \Ket{0} = -\Ket{\beta, \alpha}
\end{equation}
%
Surprisingly, the anti-symmetry property is automatically ensured!
With second quantization, many-electron states could
be written out with no pain.

For a one-electron state,
\begin{equation}
\Ket{\alpha} = c_\alpha^\dagger \Ket{0}
\end{equation}
%
For a two-electron state,
\begin{equation}
\Ket{\alpha,\beta} = c_\beta^\dagger c_\alpha^\dagger \Ket{0}
\end{equation}
%
For a three-electron state,
\begin{equation}
\Ket{\alpha,\beta,\gamma} = c_\gamma^\dagger c_\beta^\dagger c_\alpha^\dagger \Ket{0}
\end{equation}
%
We do not need to worry about the anti-symmetry, because it is taken care by
those operators automatically. This is the idea of second quantization.
It must be pointed out, second quantization does not involve any new physics.
Sometimes this name is misleading that people tend to ask, ``Wait, what
was the first quantization? Well, if the quantization of electron was the first,
what is the second one?'' No, no! Nothing is further quantized. Second quantization
is just a novel ``algebra'' that simplifies the formalism of many-body problems.

Suppose we have a state
\begin{equation}
\Ket{\text{example}} = \frac{1}{\sqrt{6}} \left( c_\alpha^\dagger c_\beta^\dagger c_\gamma^\dagger +2c_\delta^\dagger c_\epsilon^\dagger c_\zeta^\dagger +c_\eta^\dagger c_\theta^\dagger c_\iota^\dagger \right)|0\rangle
\end{equation}
%
which is a linear combination of three different Slater determinants.
It would be horrible to express it in real space (repeat Eqn.~(\ref{eq:3SD})
three times). Second quantization provides us a very convenient way
to handle many-body states.

\section{Creation and annihilation operators}
We start from the vacuum state $\Ket{0}$, which is a state with no electron.
Although without electron, it is defined to be normalized $\Braket{0|0}=1$.
Next, we introduce the creation operator $c_\alpha^\dagger$.
If $c_\alpha^\dagger$ applies on a vacuum state,
it creates one electron with state $\alpha$,
\begin{equation}
c_\alpha^\dagger \Ket{0} = \Ket{\alpha}
\end{equation}
%
``Hum? Create an electron out of vacuum?'' No, no! We are not going to
set up a lab to create electrons out of photons or whatever. This is purely
an algebra. By saying ``create'', it is from a mathematics point of view,
not a physical process. Similarly, we have an electron annihilation operator
$c_\alpha$. If $c_\alpha$ applies on a vacuum state, it returns zero
\begin{equation}
c_\alpha \Ket{0} = 0
\end{equation}
%
Previously, we claimed that the creation operators anti-commute:
$c_\alpha^\dagger c_\beta^\dagger = -c_\beta^\dagger c_\alpha^\dagger$.
This is one of the definitions in second quantization. Now, the commutation
relation between a creator and an annihilator is defined as
\begin{equation}
c_\alpha c_\beta^\dagger = \Braket{\alpha|\beta} -c_\beta^\dagger c_\alpha
\end{equation}
%
When working with orthonormal basis states, $\Braket{\alpha|\beta}=\delta_{\alpha\beta}$.
Let's see what happens if $c_\alpha$ applies on a state $\Ket{\alpha}$
\begin{equation}
c_\alpha \Ket{\alpha} = c_\alpha c_\alpha^\dagger \Ket{0}
= (1 - c_\alpha^\dagger c_\alpha) \Ket{0}
= \Ket{0} - c_\alpha^\dagger \underbrace{c_\alpha \Ket{0}}_{=0} = \Ket{0}
\end{equation}
%
As the name suggests, it removes one electron from $\Ket{\alpha}$ and
brings back the vacuum state. But this is purely an algebraic consequence,
not a definition.
The entire definition of second quantization
algebra are summarized in Table~\ref{table:secondQ}.
%
\begin{table}[h!]
\caption{The definition of second quantization algebra.}
\label{table:secondQ}
\begin{equation*}
\renewcommand\arraystretch{1.8}
\begin{array}{|>{\displaystyle}r >{\displaystyle}c >{\displaystyle}l|}
  \hline
  \Braket{0|0} & = & 1 \\
  c_\alpha\Ket{0} & = & 0 \\
  \{c_\alpha^\dagger, c_\beta^\dagger\} & = & 0 \\
  \{c_\alpha, c_\beta\} & = & 0 \\
  \{c_\alpha, c_\beta^\dagger\} & = & \Braket{\alpha|\beta} \\[0.2em]
  \hline
\end{array}
\end{equation*}
\end{table}

where the anti-commutator,
\begin{equation}
\{A,B\} \equiv AB + BA
\end{equation}
%
Believe it or not, with simply five definitions,
Table~\ref{table:secondQ} defines the complete system which formulates
second quantization.

\section{The bridge between first and second quantization}
A two-electron Slater determinant in first quantization (real space),
\begin{equation*}
\frac{1}{\sqrt{2}}(\varphi_\alpha(\vec{r}_1)\varphi_\beta(\vec{r}_2)
- \varphi_\beta(\vec{r}_1)\varphi_\alpha(\vec{r}_2))
\end{equation*}
%
A two-electron Slater determinant in second quantization (configuration space),
\begin{equation*}
c_\beta^\dagger c_\alpha^\dagger \Ket{0}
\end{equation*}
%
However, these two are not the same:
\begin{equation} \label{eq:real2nd}
c_\beta^\dagger c_\alpha^\dagger \Ket{0} \ne
\frac{1}{\sqrt{2}}(\varphi_\alpha(\vec{r}_1)\varphi_\beta(\vec{r}_2)
- \varphi_\beta(\vec{r}_1)\varphi_\alpha(\vec{r}_2))
\end{equation}
%
A wave function is a wave function and a state is a state.
They describe the same Slater determinant, but one cannot put an equal sign in between.
To make the connection between real space and second quantization, we need some
special electron creators and annihilators (called field operators).
Although physically not possible,
algebraically we can ``create'' an electron in such a state that it is
exactly at position $\vec{r}$. We denote
$c_\vec{r}^\dagger \Ket{0} = \Ket{\vec{r}}$.\footnote{Because of the importance of
these special creators and annihilators, they get a name, field operators. A more
standard way to write field operators are $\hat{\Psi}(\vec{r})$ and
$\hat{\Psi}^\dagger(\vec{r})$ (see Reference \cite{GBK}). But I would like to 
stick with $c_\vec{r}$ and $c_\vec{r}^\dagger$ to simplify our notations.}
Suppose we have an $\Ket{\alpha}$ state which in real space corresponds
to wave function $\varphi_\alpha(\vec{r})$.
Considering $\varphi_\alpha(\vec{r})$ as an amplitude, $c_\alpha^\dagger$
and $c_\vec{r}^\dagger$ are (intuitively) related as
\begin{equation} \label{eq:cacr}
c_\alpha^\dagger = \int d^3r\,\varphi_\alpha(\vec{r}) c_\vec{r}^\dagger
\end{equation}
%
Conversely, if we have a complete(!) set of single electron wave
functions $\varphi_{\alpha_n}(\vec{r})$, we can expand the field operators
in terms of the corresponding creators and annihilators
\begin{equation} \label{eq:crca}
c_\vec{r}^\dagger = \sum_n \varphi_{\alpha_n}(\vec{r}) c_{\alpha_n}^\dagger
\end{equation}
%
Using Eqn.~(\ref{eq:cacr}), we find the anti-commutation relation
\begin{equation}
\{c_\vec{r}, c_\alpha^\dagger \} =
\int d^3r'\,\varphi_\alpha(\vec{r'}) \{c_\vec{r}, c_\vec{r'}^\dagger \} = \varphi_\alpha(\vec{r})
\end{equation}
which is such a golden relation that helps us bridge
second quantization to real space. For example,
a one-electron Slater determinant,
\begin{equation}
\Braket{\vec{r}_1|\alpha} = \Bra{0} c_{\vec{r}_1} c_{\alpha}^\dagger \Ket{0}
= \Bra{0} \varphi_{\alpha}(\vec{r}_1) - c_{\alpha}^\dagger c_{\vec{r}_1} \Ket{0}
= \varphi_{\alpha}(\vec{r}_1)
\end{equation}
Nice! We get back our wave function in real space. Next,
for a two-electron Slater determinant,
\begin{align}
\Braket{\vec{r}_2,\vec{r}_1|\alpha,\beta}
& = \Bra{0} c_{\vec{r}_1} c_{\vec{r}_2} c_{\beta}^\dagger c_{\alpha}^\dagger \Ket{0} \nonumber \\
& = \Bra{0} c_{\vec{r}_1} (\varphi_\beta(\vec{r_2}) - c_{\beta}^\dagger c_{\vec{r}_2}) c_{\alpha}^\dagger \Ket{0} \nonumber \\
& = \Bra{0} c_{\vec{r}_1} c_\alpha^\dagger \Ket{0} \varphi_\beta(\vec{r}_2) - \Bra{0} c_{\vec{r}_1} c_{\beta}^\dagger c_{\vec{r}_2} c_\alpha^\dagger \Ket{0} \nonumber \\
& = \varphi_\alpha(\vec{r}_1) \varphi_\beta(\vec{r}_2) - \varphi_\beta(\vec{r}_1) \varphi_\alpha(\vec{r}_2)
\end{align}
%
Impressive! Even the two-electron anti-symmetric wave function is automatically returned.
A proof by induction is nicely discussed in Reference \cite{GBK}. Here we quote
the conclusion, for an $N$-electron state, its real-space Slater determinant
representation is given by
\begin{equation} \label{eq:bridge}
\boxed{
\Phi_{\alpha_1\cdots\alpha_N}(\vec{r}_1,\ldots,\vec{r}_N)
= \frac{1}{\sqrt{N!}}
\Bra{0} c_{\vec{r}_1} c_{\vec{r}_2} \cdots c_{\vec{r}_N}
c_{\alpha_N}^\dagger \cdots c_{\alpha_2}^\dagger c_{\alpha_1}^\dagger \Ket{0}
}
\end{equation}
%
From our quantum mechanics lectures, we often see the relation,
\begin{equation}
\Braket{\alpha|\beta} = \int d^3r\, \conj{\varphi_\alpha}(\vec{r}) \varphi_\beta(\vec{r})
\end{equation}
This can also be shown using our field operators:
\begin{align}
\Braket{\alpha|\beta}
& = \Bra{0} c_\alpha c_\beta^\dagger \Ket{0}
 = \Bra{0} \int d^3r\, \conj{\varphi_\alpha}(\vec{r}) c_\vec{r} \varphi_\beta(\vec{r}) c_{\vec{r}}^\dagger \Ket{0} \nonumber \\
& = \int d^3r\, \conj{\varphi_\alpha}(\vec{r}) \varphi_\beta(\vec{r}) \underbrace{\Bra{0} c_\vec{r} c_{\vec{r}}^\dagger \Ket{0}}_{=1}
= \int d^3r\, \conj{\varphi_\alpha}(\vec{r}) \varphi_\beta(\vec{r})
\end{align}
Similarly, for the two electron case,
\begin{align}
\Braket{\alpha,\beta|\gamma,\delta}
& = \Bra{0} c_\beta c_\alpha c_\delta^\dagger c_\gamma^\dagger \Ket{0}
= \Bra{0} c_\alpha c_\beta c_\gamma^\dagger c_\delta^\dagger \Ket{0} \nonumber \\
& = \Bra{0} \int d^3r_1\, \conj{\varphi_\alpha}(\vec{r}_1) c_{\vec{r}_1} \varphi_\delta(\vec{r}_1) c_{\vec{r}_1}^\dagger
\int d^3r_2\, \conj{\varphi_\beta}(\vec{r}_2) c_{\vec{r}_2} \varphi_\gamma(\vec{r}_2) c_{\vec{r}_2}^\dagger \Ket{0} \nonumber \\
& = \int d^3r_1\int d^3r_2\, \conj{\varphi_\alpha}(\vec{r}_1) \conj{\varphi_\beta}(\vec{r}_2) \varphi_\gamma(\vec{r}_2) \varphi_\delta(\vec{r}_1) \underbrace{\Bra{0} c_{\vec{r}_1} c_{\vec{r}_1}^\dagger c_{\vec{r}_2} c_{\vec{r}_2}^\dagger \Ket{0}}_{=1}
\end{align}
Those lovely operators $c_{\vec{r}}$ and $c_{\vec{r}}^\dagger$ play a role
bridging first and second quantization. But they never appear explicitly
in either first or second quantization!

\section{Representation of $n$-body operators}
In Chapter~\ref{ch:4}, we were working with the Coulomb repulsion Hamiltonian,
\begin{equation} \label{eq:appHu}
H_U = \sum_{i<j}^N \frac{1}{|\vec{r}_i - \vec{r}_j|}
\end{equation}
which is a two-body operator.

In Chapter~\ref{ch:6}, we introduced the spin-orbit coupling Hamiltonian,
\begin{equation} \label{eq:appHso}
H_\text{SO} = \sum_{i=1}^N \xi(r_i) \boldsymbol{\ell}_i\cdot\vec{s}_i
\end{equation}
which is a one-body operator.

Eqn.~(\ref{eq:appHu}) and Eqn.~(\ref{eq:appHso}) are in the form of
the so called first quantization. They operate on real-space wave functions.
A second quantization many-body state is, however, not compatible with
those operators. A beautiful discussion (you must give a look)
of transforming real-space operators to second quantization operators
is given in \cite{GBK}. A key idea is to use the ``bridge'' in Eqn.~(\ref{eq:bridge}).
To avoid repeating the same content, I write down the results directly:

For the Coulomb repulsion Hamiltonian,
\begin{equation}
H_U = \frac{1}{2} \sum_{\alpha,\beta,\gamma,\delta} \Bra{\alpha,\beta} \frac{1}{|\vec{r}_1-\vec{r}_2|} \Ket{\gamma,\delta} c_\alpha^\dag c_\beta^\dag c_\gamma c_\delta
\end{equation}
which is given in Eqn.~(\ref{eq:U2nd}).

For the spin-orbit coupling Hamiltonian,
\begin{equation}
H_\text{SO} = \sum_{\alpha,\beta} \Bra{\alpha} \xi(r) \boldsymbol{\ell}\cdot\vec{s} \Ket{\beta} c_\alpha^\dagger c_\beta
\end{equation}
which is given in Eqn.~(\ref{eq:SO2nd}).

They are the Hamiltonians compatible with second quantization states.

\section{Electron-hole transformation}
In this section, we would like to restrict our discussion on atomic
shell basis states instead of general states.
In Eqn.~(\ref{eq:createorder}), we made a convention that for a
fully occupied shell, the electron creators
are arranged in the following way:
\vspace{-0.5em}
\begin{equation}
\begin{array}{c|c|c|c|}
\multicolumn{1}{c}{} & \multicolumn{1}{c}{\phantom{,}1\phantom{,}} & \multicolumn{1}{c}{\phantom{,}0\phantom{,}} & \multicolumn{1}{c}{-1} \\ \cline{2-4}
\uparrow & \bullet & \bullet & \bullet \\ \cline{2-4}
\downarrow & \bullet & \bullet & \bullet \\
\cline{2-4}
\multicolumn{1}{c}{}
\end{array} =
c_{-1\downarrow}^\dagger c_{0\downarrow}^\dagger c_{1\downarrow}^\dagger
c_{-1\uparrow}^\dagger c_{0\uparrow}^\dagger c_{1\uparrow}^\dagger
\Ket{0}
\end{equation}

\vspace{-2em}
Now we understand why it is important to make such a convention: the
convention is arbitrary, but once it is decided, it must remain
unchanged through the entire discussion, since changing the order of
electron creators involves fermi-signs ($\pm1$).

To motivate the topic of this section, let's
consider an almost-full shell, say, a $d^8$:
\vspace{-0.5em}
\begin{equation*}
\begin{array}{c|c|c|c|c|c|}
\multicolumn{1}{c}{} & \multicolumn{1}{c}{\phantom{,}2\phantom{,}} &
\multicolumn{1}{c}{\phantom{,}1\phantom{,}} & \multicolumn{1}{c}{\phantom{,}0\phantom{,}} &
\multicolumn{1}{c}{-1} & \multicolumn{1}{c}{-2} \\ \cline{2-6}
\uparrow & \bullet & \bullet & \bullet & \bullet &  \\ \cline{2-6}
\downarrow & \bullet & \bullet &  & \bullet & \bullet \\
\cline{2-6}
\end{array}
\end{equation*}
%
We would write it in terms of electron creation operators as
\begin{equation*}
c_{-2\downarrow}^\dagger c_{-1\downarrow}^\dagger c_{1\downarrow}^\dagger c_{2\downarrow}^\dagger
c_{-1\uparrow}^\dagger c_{0\uparrow}^\dagger c_{1\uparrow}^\dagger c_{2\uparrow}^\dagger
\Ket{0}
\end{equation*}
%
This becomes a bit cumbersome and not very readable (but of course
much simpler than its real-space form). We noticed that if we express
the same state in terms of the unoccupied sites, the expression will become
much shorter. What we need to do is to transform our ``electron algebra''
into a ``hole algebra''.
Let's start from the fully occupied $d$ shell
\begin{equation} \label{eq:full}
\Ket{\text{full}} = 
c_{-2\downarrow}^\dagger c_{-1\downarrow}^\dagger c_{0\downarrow}^\dagger c_{1\downarrow}^\dagger c_{2\downarrow}^\dagger
c_{-2\uparrow}^\dagger c_{-1\uparrow}^\dagger c_{0\uparrow}^\dagger c_{1\uparrow}^\dagger c_{2\uparrow}^\dagger
\Ket{0}
\end{equation}
%
Notice that a $\Ket{\text{full}}$ state also has $M_L=0$ and $M_S=0$.
From the hole's point of view, the $\Ket{\text{full}}$ state behaves like a
``vacuum'' state. Creating a hole at site ($-m,-\sigma$) on a $\Ket{\text{full}}$ state
leaves the system with momentum $M_L=m$ and $M_S=\sigma$.
Hence we could define our hole creation operator as
\begin{equation} \label{eq:simpdef}
h_{m\sigma}^\dagger = c_{-m,-\sigma}
\end{equation}
%
But this is not very convenient. Because what we really want is, for example,
\begin{equation} \label{eq:hfull}
\begin{array}{|c|c|c|c|c|}
\hline
\bullet & \bullet & \bullet & \bullet &  \\ \hline
\bullet & \bullet & \bullet & \bullet & \bullet \\
\hline
\end{array}
= h_{2\downarrow}^\dagger \Ket{\text{full}}
\end{equation}

\vspace{-1em}
However,
\begin{align}
\begin{array}{|c|c|c|c|c|}
\hline
\bullet & \bullet & \bullet & \bullet &  \\ \hline
\bullet & \bullet & \bullet & \bullet & \bullet \\
\hline
\end{array}
& = c_{-2\downarrow}^\dagger c_{-1\downarrow}^\dagger c_{0\downarrow}^\dagger c_{1\downarrow}^\dagger c_{2\downarrow}^\dagger
c_{-1\uparrow}^\dagger c_{0\uparrow}^\dagger c_{1\uparrow}^\dagger c_{2\uparrow}^\dagger
\Ket{0} \nonumber \\
& = c_{-2\downarrow}^\dagger c_{-1\downarrow}^\dagger c_{0\downarrow}^\dagger c_{1\downarrow}^\dagger c_{2\downarrow}^\dagger
\boxed{c_{-2\uparrow}} c_{-2\uparrow}^\dagger c_{-1\uparrow}^\dagger c_{0\uparrow}^\dagger c_{1\uparrow}^\dagger c_{2\uparrow}^\dagger
\Ket{0} \nonumber \\
& = (-1)^{5} \boxed{c_{-2\uparrow}}
\underbrace{c_{-2\downarrow}^\dagger c_{-1\downarrow}^\dagger c_{0\downarrow}^\dagger c_{1\downarrow}^\dagger c_{2\downarrow}^\dagger
c_{-2\uparrow}^\dagger c_{-1\uparrow}^\dagger c_{0\uparrow}^\dagger c_{1\uparrow}^\dagger c_{2\uparrow}^\dagger
\Ket{0}}_{\Ket{\text{full}}} \nonumber \\
& = - c_{-2\uparrow} \Ket{\text{full}} = - h_{2\downarrow}^\dagger \Ket{\text{full}}
\end{align}
%
If we really want to write as the way in Eqn.~(\ref{eq:hfull}),
we must absorb the fermi-sign into definition (\ref{eq:simpdef}).
According to our full shell definition, this fermi-sign has the following pattern
\begin{equation*}
\begin{array}{|c|}
\hline
- \\ \hline
+ \\
\hline
\end{array}
\quad
\begin{array}{|c|c|c|}
\hline
- & + & - \\ \hline
+ & - & + \\
\hline
\end{array}
\quad
\begin{array}{|c|c|c|c|c|}
\hline
- & + & - & + & - \\ \hline
+ & - & + & - & + \\
\hline
\end{array}
\quad
\begin{array}{|c|c|c|c|c|c|c|}
\hline
- & + & - & + & - & + & - \\ \hline
+ & - & + & - & + & - & + \\
\hline
\end{array}
\end{equation*}
%
Therefore, we define, (the definition is subject to how a $\Ket{\text{full}}$ is defined)
\begin{equation} \label{eq:ehtrans}
\boxed{
h_{m\sigma}^\dagger = (-1)^{l+m+\sigma-\frac{1}{2}} c_{-m,-\sigma}
}
\end{equation}
%
The next question is how to arrange these hole operators. Previously
we made a convention of ordering electron creators. Now we no longer
have this freedom to define new convention of ordering
hole creators. As a consequence from previous convention, the hole
creators should be ordered in the following way:
(notice that it is the same order of putting electrons)
\vspace{-0.5em}
\begin{equation}
\begin{array}{c|c|c|c|}
\multicolumn{1}{c}{} & \multicolumn{1}{c}{\phantom{,}1\phantom{,}} & \multicolumn{1}{c}{\phantom{,}0\phantom{,}} & \multicolumn{1}{c}{-1} \\ \cline{2-4}
\uparrow &  &  &  \\ \cline{2-4}
\downarrow &  &  &  \\
\cline{2-4}
\multicolumn{1}{c}{}
\end{array} =
h_{1\uparrow}^\dagger h_{0\uparrow}^\dagger h_{-1\uparrow}^\dagger
h_{1\downarrow}^\dagger h_{0\downarrow}^\dagger h_{-1\downarrow}^\dagger
\Ket{\text{full}}
\end{equation}

\vspace{-2em}
In general, an $N$-electron basis vector and a $(2(2l-1)-N)$-hole basis vector
are equivalent by the relation,
\begin{equation*}
\prod_{i=1}^N c_{m_i\sigma_i}^\dagger \Ket{0} =
\prod_{j=1}^{2(2l-1)-N} h_{m_j\sigma_j}^\dagger \Ket{\text{full}}
\end{equation*}
The indices $\{j\}$ run over the complement part of indices $\{i\}$,
for example,
\begin{equation}
\begin{array}{|c|c|c|c|c|}
\hline
i_1 & i_2 & i_3 & i_4 & j_1 \\ \hline
i_5 & i_6 & j_2 & i_7 & i_8 \\
\hline
\end{array}
\end{equation}
To show their equivalence,
\begin{equation}
\prod_{j=1}^{2(2l-1)-N} h_{m_j\sigma_j}^\dagger \Ket{\text{full}}
= \prod_{j=1}^{2(2l-1)-N} (-1)^{l+m_j+\sigma_j-\frac{1}{2}} c_{-m_j,-\sigma_j} \prod_{i=1}^{2(2l+1)} c_{m_i\sigma_i}^\dagger \Ket{0}
\end{equation}
But anti-commuting $c_{-m_j,-\sigma_j}$ into the full shell always cancels the
$(-1)^{l+m_j+\sigma_j-\frac{1}{2}}$ factor (that is how this factor is designed for).
After all anti-commutations, what left is,
\begin{equation}
\prod_{j=1}^{2(2l-1)-N} h_{m_j\sigma_j}^\dagger \Ket{\text{full}}
= \prod_{i=1}^N c_{m_i\sigma_i}^\dagger \Ket{0}
\end{equation}
%
We can verify that our example state
\begin{align}
\begin{array}{|c|c|c|c|c|}
\hline
\bullet & \bullet & \bullet & \bullet &  \\ \hline
\bullet & \bullet &  & \bullet & \bullet \\
\hline
\end{array}
& = h_{0\uparrow}^\dagger h_{2\downarrow}^\dagger \Ket{\text{full}}
= - c_{0\downarrow} c_{-2\uparrow} \Ket{\text{full}} \nonumber \\
& = - c_{0\downarrow} \boxed{c_{-2\uparrow}}
c_{-2\downarrow}^\dagger c_{-1\downarrow}^\dagger c_{0\downarrow}^\dagger c_{1\downarrow}^\dagger c_{2\downarrow}^\dagger
c_{-2\uparrow}^\dagger c_{-1\uparrow}^\dagger c_{0\uparrow}^\dagger c_{1\uparrow}^\dagger c_{2\uparrow}^\dagger
\Ket{0} \nonumber \\
& = (-1)^6 c_{0\downarrow}
c_{-2\downarrow}^\dagger c_{-1\downarrow}^\dagger c_{0\downarrow}^\dagger c_{1\downarrow}^\dagger c_{2\downarrow}^\dagger
\boxed{c_{-2\uparrow}} c_{-2\uparrow}^\dagger c_{-1\uparrow}^\dagger c_{0\uparrow}^\dagger c_{1\uparrow}^\dagger c_{2\uparrow}^\dagger
\Ket{0} \nonumber \\
& = (-1)^6 \boxed{c_{0\downarrow}}
c_{-2\downarrow}^\dagger c_{-1\downarrow}^\dagger c_{0\downarrow}^\dagger c_{1\downarrow}^\dagger c_{2\downarrow}^\dagger
c_{-1\uparrow}^\dagger c_{0\uparrow}^\dagger c_{1\uparrow}^\dagger c_{2\uparrow}^\dagger
\Ket{0} \nonumber \\
& = (-1)^8
c_{-2\downarrow}^\dagger c_{-1\downarrow}^\dagger \boxed{c_{0\downarrow}} c_{0\downarrow}^\dagger c_{1\downarrow}^\dagger c_{2\downarrow}^\dagger
c_{-1\uparrow}^\dagger c_{0\uparrow}^\dagger c_{1\uparrow}^\dagger c_{2\uparrow}^\dagger
\Ket{0} \nonumber \\
& =
c_{-2\downarrow}^\dagger c_{-1\downarrow}^\dagger c_{1\downarrow}^\dagger c_{2\downarrow}^\dagger
c_{-1\uparrow}^\dagger c_{0\uparrow}^\dagger c_{1\uparrow}^\dagger c_{2\uparrow}^\dagger
\Ket{0}
\end{align}
is indeed equivalent to the expression in terms of electron creators.

Eqn.~(\ref{eq:ehtrans}) is essentially the portal connecting the electron world with
the hole world. With this definition, we can investigate the properties of hole
operators. First of all, the $\Ket{\text{full}}$ state is normalized,
$\Braket{\text{full}|\text{full}}=1$. If we apply a hole annihilation operator
on it, according to Pauli exclusion principle, $h_{m\sigma}\Ket{\text{full}}=0$.
Now, we test the anti-commutation relation between hole operators.
\begin{align}
\{ h_{m_1\sigma_1}^\dagger, h_{m_2\sigma_2}^\dagger \}
& = (-1)^{2l+m_1+m_2+\sigma_1+\sigma_2-1} \{ c_{-m_1,-\sigma_1}, c_{-m_2,-\sigma_2} \} = 0 \\
\{ h_{m_1\sigma_1}, h_{m_2\sigma_2} \}
& = (-1)^{2l+m_1+m_2+\sigma_1+\sigma_2-1} \{ c_{-m_1,-\sigma_1}^\dagger, c_{-m_2,-\sigma_2}^\dagger \} = 0 \\
\{ h_{m_1\sigma_1}, h_{m_2\sigma_2}^\dagger \}
& = (-1)^{2l+m_1+m_2+\sigma_1+\sigma_2-1} \{ c_{-m_1,-\sigma_1}^\dagger, c_{-m_2,-\sigma_2} \} \nonumber \\
& = (-1)^{2l+m_1+m_2+\sigma_1+\sigma_2-1} \delta_{m_1m_2} \delta_{\sigma_1\sigma_2}
= \delta_{m_1m_2} \delta_{\sigma_1\sigma_2}
\end{align}
%
It turns out (not by definition), our hole operators behave exactly as electron
operators.
\begin{table}[h!]
\caption{Second quantization algebra from the hole's perspective.}
\label{table:secondQhole}
\begin{equation*}
\renewcommand\arraystretch{1.8}
\begin{array}{|>{\displaystyle}r >{\displaystyle}c >{\displaystyle}l|}
  \hline
  \Braket{\text{full}|\text{full}} & = & 1 \\
  h_{m\sigma}\Ket{\text{full}} & = & 0 \\
  \{h_{m_1\sigma_1}^\dagger, h_{m_2\sigma_2}^\dagger\} & = & 0 \\
  \{h_{m_1\sigma_1}, h_{m_2\sigma_2}\} & = & 0 \\
  \{h_{m_1\sigma_1}, h_{m_2\sigma_2}^\dagger\} & = & \delta_{m_1m_2} \delta_{\sigma_1\sigma_2} \\[0.2em]
  \hline
\end{array}
\end{equation*}
\end{table}


